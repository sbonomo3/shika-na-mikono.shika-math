\chapter{Form I-IV Topics and Activities}

\section{Form I}

	\subsection{Numbers (I)} \label{numbers}
	\begin{itemize}
	\item\nameref{flashcards} can be used to illustrate base ten numeration, as well as reinforce simple arithmetic. Use them to play \nameref{games} such as \nameref{aroundtheworld} to help students master their times tables.
	\item Use a \nameref{numberline} to show addition and subtraction of integers. Using students as the number line is a great way to involve them in the lesson. \nameref{toka} is a great game for students to practice using a number line.
	\item Operations on integers (BODMAS) can be taught using games such as \nameref{fourfours} and \nameref{24squares}.
	\item Be conscious of problem-solving methods common in Tanzanian primary schools for topics such as GCF and LCM which may be instilled in the students. 
	\end{itemize}
	
	\subsection{Fractions} \label{fractions}
	\begin{itemize}
	\item See the Hands-on Tools section for how to make \nameref{fracsanddecs}.
	\item Represent fractions using bottles, sticks, colored circles, etc.
	\item Have a class-wide ``Mock Market'', where students buy and sell fractions of quantities of items such as rice, flour, oil, milk, etc. Can they adjust the prices according to the fractions purchased?
	\item See more \nameref{games} such as \nameref{memory}, \nameref{20questions}, \nameref{guesswho} and \nameref{snap}, as well as \nameref{classacts} on \nameref{classactsfracsdecs}.
	\end{itemize}
	
	\subsection{Decimals and Percentages} \label{decspercents}
	\begin{itemize}
	\item See the Hands-on Tools section for how to make \nameref{fracsanddecs}.
	\item Create movable numbers on manila paper to show that decimal points must be lined up for addition and subtraction.
	\item Play a decimals and percentages \nameref{bingo} game. The students fill their cards with numbers in one form (e.g. decimals) and the teacher calls out numbers in another form (e.g. percentages). Students must first convert the number before being able to cover spaces on their boards. Play this game with fractions as well.
	\item Have students calculate their percentage on homework or in-class assignments, based on the number of questions and how many were answered correctly. 
	\item See more \nameref{games} such as \nameref{memory}, \nameref{20questions}, \nameref{guesswho} and \nameref{snap}, as well as \nameref{classacts} on \nameref{classactsfracsdecs}.
	\end{itemize}
	
	\subsection{Units}
	\begin{itemize}
	\item Use tape measures, rulers and metre rulers to show conversions in length.
	\item Use various sized bottles to demonstrate capacity. Show that three 500 mL bottles is the same volume as a single 1.5 L bottle to reinforce multiplication of fractions.
	\item Students can create a clock to learn units of time.
	\end{itemize}
	
	\subsection{Approximations}
	\begin{itemize}
	\item \nameref{flashcards} can be used to create movable numbers for teaching significant figures.
	\item Have students guess how many pieces of candy are in a small jar. The student who gets closest to the correct number wins the candy!
	\item Estimate the number of students in each stream of every form to approximate the total students at school. Compare this to the actual total. Is it a good estimate?
	\item Ask students to approximate how much corn can be yielded from an acre of farmland. How many acres would be needed for a certain desired profit?
	\end{itemize}
	
	\subsection{Geometry} \label{geometry}
	\begin{itemize}
	\item Use a \nameref{geoboard} to create lines and various polygons and angles.
	\item Circles can be drawn on the ground using a string radius. Students can construct their own circles by making their own \nameref{compass}.
	\item Students can create their own \nameref{protractor} to construct various angles.
	\item Salama says: students act out points, lines and angles using their arms.
	\item Cut out a large triangle having any dimensions. Label each angle and then cut out and tape them together on the board to show that they form a straight line and thus add up to $180^\circ$. Repeat for different polygons to discover formulas for finding interior and exterior angles.
	\item See \nameref{games} such as \nameref{aroundtheworld} and \nameref{battleship}, as well as \nameref{classacts} on \nameref{classactsgeo}.
		\end{itemize}
	
	\subsection{Algebra} \label{algebra}
	\begin{itemize}
	\item Bring a bucket of apples and bananas. Apples represent the ``a'' and bananas represent ``b''. Unlike
objects cannot be added.
	\item Colored chalk can be used. Red and blue cannot be added together.
	\item Use a scale to represent both sides of the equation. If something is added or subtracted to one side, it must be done the same to the other side to maintain balance.
	\item Use cards by writing the variable on one side and its value on the other. When the equation has been solved, the variable reveals its identity by flipping over.
	\item See more ideas in \nameref{classacts} on \nameref{classactsalgebra}.
	\end{itemize}
	
	\subsection{Numbers (II)}
	\begin{itemize}
	\item A \nameref{numberline} is the best tool for teaching numbers, even fractions that are rational numbers.
	\end{itemize}
	
	\subsection{Ratio, Profit and Loss}
	\begin{itemize}
	\item Ratio can be taught with objects to show that Juma gets 2 items for every 3 items that Mary receives.
	\item Show profit and loss through a bottle cap business demonstration, for a given buying price and selling price per bottle cap.
	\item Have students investigate ratios of different body parts. See more in \nameref{classacts} on \nameref{classactsratio}.
	\end{itemize}
	
	\subsection{Coordinate Geometry}
	\begin{itemize}
	\item See the section on Form IV \nameref{coordgeo}
	\end{itemize}
	
	\subsection{Perimeter and Area}
	\begin{itemize}
	\item See the section on Form IV \nameref{areaper}
	\end{itemize}

\section{Form II}

	\subsection{Exponents and Radicals}
	\begin{itemize}
	\item Use \nameref{flashcards} with ``x'' written on them to show that multiplying four of these is equal to ``x'' to the fourth power. 
	\item Add and remove cards in the numerator and denominator to demonstrate the laws of exponents in this way.
	\item Guide students to make tables or posters of the laws of exponents to be posted around the classroom.
	\end{itemize}
	
	\subsection{Algebra}
	\begin{itemize}
	\item Use colored chalk to teach BODMAS.
	\item Factor trees can be helpful when teaching factorization.
	\end{itemize}
	
	\subsection{Quadratic Equations}
	\begin{itemize}
	\item Use crazy names for variables so that students do not combine different variables.
	\item Use a song to help students remember the quadratic formula.
	\end{itemize}
	
	\subsection{Logarithms}
	\begin{itemize}
	\item Help students create charts of the laws of logarithms. These can be hung around the classroom for easy reference.
	\item Ensure students get some practice using four figure tables to solve problems as this is often required on the NECTA exams.
	\end{itemize}
	
	\subsection{Congruence} \label{congruence}
	\begin{itemize}
	\item Draw pairs of various congruent shapes on cards and play \nameref{games} such as \nameref{memory} and \nameref{snap} to get students to identify congruent pairs.
	\item Use origami paper to show how folding squares can produce different congruent shapes.
	\end{itemize}
	
	\subsection{Similarity} \label{similarity}
	\begin{itemize}
	\item Use \nameref{games} such as \nameref{memory} and \nameref{snap} to help students identify and pair off similar figures.
	\item Construct a 2-D house out of cut out triangles and rectangles. Then have students cut out a set of shapes that are similar to the ones you used, and another set which are not similar. Which house better resembles the one you made?
	\end{itemize}
	
	\subsection{Geometrical Transformations}
	\begin{itemize}
	\item Use a \nameref{geoboard} or an x-y plane in the ground to show rotation, reflections, etc.
	\item Students can be used to demonstrate rotation, reflection and translation.
	\item Use bottle caps to show original and transformed shapes side-by-side. See more in \nameref{classacts} on \nameref{classactscoordgeo}.
	\end{itemize}
	
	\subsection{Pythagoras Theorem}
	\begin{itemize}
	\item Help students derive Pythagoras Theorem by using a right triangle made with squares acting as each side. Use the areas of the squares to show that $a^2 + b^2 = c^2$.
	\item Demonstrate a proof of Pythagoras Theorem using a  square sheet of paper folded so that each side is split into lengths of $a$ and $b$, creating an inner square of side length $c$.
	\end{itemize}
	
	\subsection{Trigonometry}
	\begin{itemize}
	\item See the section on Form IV \nameref{trig}.
	\end{itemize}
	
	\subsection{Sets}
	\begin{itemize}
	\item Construct a tangible Venn diagram to place movable shapes, colored objects, etc.
	\item Demonstrate sets and subsets using playing cards, \nameref{dominoes}, dinner ware or even students themselves!
	\end{itemize}
	
	\subsection{Statistics}
	\begin{itemize}
	\item See the section on Form III \nameref{statistics}.
	\end{itemize}

\section{Form III}

	\subsection{Relations} \label{relations}
	\begin{itemize}
	\item Use \nameref{dice} and \nameref{dominoes} for forming Domain and Range sets.
	\item Use a \nameref{geoboard} or \nameref{cartesianplane} to graph relations.
	\end{itemize}

	\subsection{Functions} \label{functions}
	\begin{itemize}
	\item Teach functions using the ``Function Jiko'' method:
		\begin{itemize}
		\item Functions can be likened to cooking ugali on a charcoal jiko. The necessary inputs are added to the jiko - these can be grouped together to represent the independent variable, $x$ (i.e. unga and water).
		\item Then ``stuff happends'' to the input - these are the operations on the independent variable such as multiplication, addition, etc. (i.e. stirring and heating). These must be performed in the correct order to receive the desired product.
		\item Finally, the function ``poops out'' its output after all necessary operations are performed - this is the dependent variable, $y$ (i.e. ugali).
		\item Use a real pot, spoon, etc. to demonstrate this concept to the class. Stress the fact that if you change the input, the output is affected accordingly. In a function, every input correlates to exactly one output.
		\end{itemize}
	\item Graph functions on a \nameref{geoboard} or \nameref{cartesianplane}.
	\item Play a \nameref{battleship} game to reinforce graphing different functions. See more \nameref{classacts} on \nameref{classactsfns}.
	
	\end{itemize}

	\subsection{Statistics} \label{statistics}
	\begin{itemize}
	\item Have students collect data on family or fellow students based on age, height, school level, children in the family, etc. They then organize their data in bar graphs, line graphs, pie charts, histograms, frequency distribution tables, etc. Students can also find mean, median and mode by using formulas or approximations from their graphs and tables. This is a great class project or math practical and can also be used as a way to teach the Scientific Method.
	\item Incorporate statistics on HIV\slash AIDS and malaria to increase awareness of the students in their communities.
	\item Use current football stats on well-known teams to excite and engage students.
	\item Arrange students in a line according to height or age to demonstrate median.
	\end{itemize}

	\subsection{Rates and Variations}
	\begin{itemize}
	\item Demonstrate 2:3:5 using blocks, sticks, bottles, etc. so that it will be concrete learning.
	\item Use different sized syringes or tubes to show that the rate of water flow is proportional to the size of the tube opening.
	\item Use practical examples of direct variation, such as opening and closing a faucet (bomba), cooking ugali and studying vs. test performance. Examples of indirect variation include time spent talking on the phone vs. remaining vocha, notes written vs. chalk length, etc. 
	\item Have students sketch graphs of common practical examples of variations without using numbers.
	\item Incorporate formulas from science subjects, such as chemical reaction rates, Ohm's Law, etc.
	\item Ask students to come up with examples on their own to demonstrate comprehension.
	\end{itemize}

	\subsection{Sequences and Series}
	\begin{itemize}
	\item Translate word statements into math expressions (e.g. 4th term is 10 more than the 2nd term).
	\item List many numbers in a sequence to prove the sum formulas.
	\item Use a \nameref{jeopardy}-style game to challenge students' pattern recognition using numerical and geometrical sequences. This is a great tool for developing logical thought processes.
	\item Make colored posters to hang around the classroom highlighting the distinctions between arithmetical and geometrical progressions.
	\item Have students act out sequence patterns. See more in \nameref{classacts} on \nameref{classactsseqser}.
	\end{itemize}

	\subsection{Circles} \label{circlesactivities}
	\begin{itemize}
	\item See the Hands-On Tools section for how to make \nameref{circlestools}.
	\item Play a game to have students stick paper labels to corresponding properties of circles.
	\item Use a clock to sweep out a span of time to represent a sector of a circle.
	\item Cut out circles to draw on for proving circle theorems in class.
	\item Have students draw what each theorem states to reinforce the concept.
	\end{itemize}

	\subsection{Earth as a Sphere}
	\begin{itemize}
	\item Have students make spheres out of bamboo strips. See more in Hands-On Tools for making a \nameref{globe}.
	\item Use globes, if available, or else footballs, basketballs or volleyballs to show students longitude and latitude.
	\end{itemize}

	\subsection{Accounts} \label{accountsactivities}
	\begin{itemize}
	\item Teach the principle of double entry using the ``Accounts Stendi'' method:
		\begin{itemize}
		\item Each account is like a bus stand, having an entrance (Dr.) side and an exit (Cr.) side. 
		\item When a transaction occurs between two accounts, or bus stands, the money must be transported via the Transaction Express.
		\item The bus exits the first stendi, but the driver must ``sign out'' before leaving by writing the Date of the transaction (Date of travel), Particulars (the bus's destination city\slash account), Folio (the corresponding number code for the destination) and Amount (the money being delivered) under the Cr. side (exit) of the departure stendi.
		\item When the bus reaches its destination, the driver must now ``sign in'' by writing the Date, Particulars (city\slash account of origin), Folio (number code for the origin) and Amount under the Dr. side (entrance) of the arrival stendi.
		\item This analogy helps students to see why it is necessary to record each transaction twice in the account books. For example, a Purchase must be recorded for two purposes: to account for the payment of money \emph{from the Cash Account}, and to account for the receipt of goods \emph{into the Purchases Account}.
		\item Make real accounts and a real Transaction Express bus out of manila paper to demonstrate this concept.
		\end{itemize}
	\item Use movable strips of paper as transactions and allow students to come up to the board to place them in the proper accounts.
	\item Have a small duka demonstration with the class. You have a tomato or bottle cap business and need to keep track of purchases, sales, expenses, such as transport and duka monthly rent fees, and want to know if you are making or losing money.
	\item See more in Hands-On Tools for making \nameref{accountstools}.
	\end{itemize}

\section{Form IV}

	\subsection{Coordinate Geometry} \label{coordgeo}
	\begin{itemize}
	\item A \nameref{geoboard} can be used to demonstrate slope, midpoint, parallel and perpendicular lines, etc.
	\item Allow students to create their own \nameref{cartesianplane} out of old calendars, seed bags or whatever they choose.
	\item Use \nameref{dice} to generate random coordinates for practice with plotting points and to make homework or in-class problems on slope, midpoint, distance formula, etc. Make four dice out of manila paper (2 red - one of +'s and -'s, and one of the numbers 1-6, and then 2 blue in the same way). Students roll the dice to find an x-coordinate (blue) and a y-coordinate (red) and then must plot the point on the Cartesian plane.
	\item Help students use Pythagoras Theorem to derive the distance formula.
	\item Play a \nameref{battleship} game to help students read and plot points.
	\item Have students create different shapes on a coordinate plane by plotting bottle caps. See more in \nameref{classacts} on \nameref{classactscoordgeo}.
	\end{itemize}

	\subsection{Area and Perimeter} \label{areaper}
	\begin{itemize}
	\item Use a \nameref{geoboard} to help students discover formulas of area and perimeter for squares and rectangles.
	\item Using graph paper, have students cut out rectangles having an area of 48 square units. How many different sets of dimensions be used to generate this area?
	\item \nameref{circlestools}, such as bike wheels, can be used to show circumference and also to measure the perimeter around the school or a classroom.
	\item Use string to measure the circumference and diameter of a circle, and to derive $\pi$.
	\item Gather several small circles or cylindrical items of various sizes. Trace the circular outlines on graph paper and for each one, count the number of squares that make up the area and the number of units of length of the radius. Tabulate the results. Can you discover a relationship between radius and area for different sized circles?
	\item Students make various polygons using a \nameref{geoboard} or cutting them out of paper, and then use the shapes to discover formulas.
	\item Have students draw polygons on graph paper (or create their own graph paper) to discover relations between areas of similar figures.
	\end{itemize}

	\subsection{Three-Dimensional Figures} \label{3dfigsactivities}
	\begin{itemize}
	\item See the Hands-On Tools section for how to make \nameref{3dfigstools}.
	\item Have a challenge to see who can identify the most 3-dimensional objects in the classroom.
	\item Use hollow containers or objects that can be opened to see interior angles and planes.
	\item Do a design challenge with students. See more in \nameref{classacts} on \nameref{classacts3d}.
	\end{itemize}

	\subsection{Probability}
	\begin{itemize}
	\item Use \nameref{dice}, playing cards, coins, bottle caps, \nameref{flashcards}, \nameref{spinners}, fruits or any objects to demonstrate probability.
	\item Have students determine the probability of selecting a boy from the class, or two boys. Does the result change if the first one goes outside before the next selection is made?
	\item Relate to genetics in Biology, or incorporate data on HIV\slash AIDS and malaria to increase awareness.
	\item Use a probability line to teach key probability terms and to apply probability to real-life situations. See more in \nameref{classacts} on \nameref{classactsprob}.
	\end{itemize}

	\subsection{Trigonometry} \label{trig}
	\begin{itemize}
	\item Guide students to develop tables of values of trig ratios for special angles for $-360^\circ \leq \theta \leq 360^\circ$ and then use them to graph $\sin$, $\cos$ and $\tan$ functions.
	\item Play a \nameref{bingo} game using trig ratios for some special angles. Instead of B-I-N-G-O, column headers are $\sin$, $\cos$ and $\tan$. Students fill their cards with various values under each trig function, and the teacher calls out a particular trig function and angle. For example, the teacher calls, ``$\sin 30^\circ$'' and students having a ``0.5'' under their $\sin$ column get to cover that space.
	\item Have students create a \nameref{clinometer} to see the application of trigonometry in determining the height of a building or tall tree.
	\end{itemize}

	\subsection{Vectors}
	\begin{itemize}
	\item Use colored origami paper cranes or cut out airplanes to use as points on a \nameref{cartesianplane}. The birds or planes start at one point and travel in a specified direction towards the destination point.
	\item Relate to applications in Physics.
	\item Use colored chalk to show that unlike components cannot be combined.
	\end{itemize}

	\subsection{Matrices and Transformations}
	\begin{itemize}
	\item Charts and colored chalk must be used to demonstrate operations on matrices.
	\item Attach a string to the origin and stretch it out to an object to show rotation about the origin.
	\item Use mirrors or spoons to show reflection.
	\item Use bottle caps to show original and transformed shapes side-by-side. See more in \nameref{classacts} on \nameref{classactscoordgeo}.
	\end{itemize}

	\subsection{Linear Programming}
	\begin{itemize}
	\item A \nameref{geoboard} can be used to demonstrate maximum and minimum.
	\item Drill students on how to generate constraints based on the word problems. Use tables to organize information. This requires the most practice as students are not familiar with many English terms.
	\item Make a math vocab list of key terms such as ``at least,'' ``at most,'' ``no fewer than,'' ``not more than,'' ``minimum'' and ``maximum.''
	\end{itemize}
