\chapter{NECTA Exam Format}
\label{cha:nectaformat}

In accordance with the revised math syllabus which went into effect in 2005, the Ministry of Education and Vocational Training has issued the following specifications regarding the format of the Basic Mathematics Form IV NECTA exam. Note that each number directly equates to the respective question number on the actual exam. Therefore, Question 11 will \emph{always} cover the topic of Linear Programming. For the Form IV NECTA exam, students have 3 hours to complete all 10 questions from Section A (worth 6 marks each) and 4 out of the 6 questions from Section B (worth 10 marks each).

\begin{center}
\begin{tabular}{cl|c|c|c|} \\ \hline
\multicolumn{1}{|c|}{\textbf{S\slash n}} & \multicolumn{1}{c|}{\textbf{Topic(s)}} & \textbf{Form(s)} & \textbf{Pts} & \textbf{Total} \\ \hline \hline
\multicolumn{5}{|c|}{\textbf{SECTION A (60 MARKS)}} \\ \hline
\multicolumn{1}{|c|}{1} & Numbers\slash Fractions\slash Decimals and Percentages\slash Approximations & I & 6 & \multirow{10}{*}{60} \\ \cline{1-4}
\multicolumn{1}{|c|}{2} & Exponents\slash Radicals\slash Logarithms & II & 6 & \\ \cline{1-4}
\multicolumn{1}{|c|}{3} & Algebra\slash Sets & II & 6 & \\ \cline{1-4}
\multicolumn{1}{|c|}{4} & Coordinate Geometry\slash Vectors & IV & 6 & \\ \cline{1-4}
\multicolumn{1}{|c|}{5} & Geometry\slash Perimeter and Area\slash Congruence\slash Similarity & I\slash II & 6 & \\ \cline{1-4}
\multicolumn{1}{|c|}{6} & Units\slash Rates and Variations & I\slash III & 6 & \\ \cline{1-4}
\multicolumn{1}{|c|}{7} & Ratio, Profit and Loss & I & 6 & \\ \cline{1-4}
\multicolumn{1}{|c|}{8} & Sequences and Series & III & 6 & \\ \cline{1-4}
\multicolumn{1}{|c|}{9} & Trigonometry\slash Pythagoras Theorem & II\slash IV & 6 & \\ \cline{1-4}
\multicolumn{1}{|c|}{10} & Quadratic Equations & II & 6 & \\ \hline
\multicolumn{5}{|c|}{\textbf{SECTION B (40 MARKS - CHOOSE 4)}} \\ \hline
\multicolumn{1}{|c|}{11} & Linear Programming & IV & 10 & \multirow{6}{*}{40} \\ \cline{1-4}
\multicolumn{1}{|c|}{12} & Statistics & III & 10 & \\ \cline{1-4}
\multicolumn{1}{|c|}{13} & Three Dimensional Figures\slash Circles\slash Earth as a Sphere & III\slash IV & 10 & \\ \cline{1-4}
\multicolumn{1}{|c|}{14} & Accounts & III & 10 & \\ \cline{1-4}
\multicolumn{1}{|c|}{15} & Matrices and Transformations & IV & 10 & \\ \cline{1-4}
\multicolumn{1}{|c|}{16} & Probability\slash Functions\slash Relations & III\slash IV & 10 & \\ \hline

\end{tabular}
\end{center}

Letter grades are assigned based on the following scale for Form IV NECTA exams:
\begin{center}
\begin{tabular}{c|c}
Score & Grade \\ \hline
81 - 100 & A \\
61 - 80 & B \\
41 - 60 & C \\
21 - 40 & D \\
0 - 20 & F \\
\end{tabular}
\end{center}

Thus, it is possible for a student to effectively pass their exam, even by knowing just \emph{two} topics very well (e.g Statistics and Accounts). Of course, it is not necessarily recommended to have students only focus on learning a few select topics, as this does not encourage thorough learning of the material, but it can be used as a motivator to students who have already written themselves off for the Math NECTA.

Have students plan ahead of time to formulate a strategy for the exam. Which 4 topics does each student feel most comfortable with from Section B? Students should have 1 fall-back topic in case one of their preferred topics is particularly challenging on an exam. Advise students to begin with problems from Section B, as these carry more marks and therefore should have more time dedicated to them. Which topics from Section A are strengths for a student, and which ones are weaknesses? Students should answer their strong topics first, and use whatever time they have remaining for weaker topics.

Planning ahead and utilizing a strategy for taking the NECTA exam will give students added confidence in their test-taking abilities and will also give them a strategy for what topics to study in most detail leading up to the exam. Make sure to give students plenty of practice exams before the actual NECTA! Many students only have one opportunity to familiarize themselves with the layout of the test during Mock Examinations a few months ahead of time, but this is not enough. Comfort and confidence in one's math abilities come from practice, practice and more practice!

