\chapter{Games and Activities for Stimulating Interest in Mathematics} \label{cha:gamesandactivities}
Math must be stimulated by puzzles, games, patterns, activities and other forms of hands-on and interactive learning. An interest in mathematical puzzles generates creative thinking and motivates students in a way that standard textbooks can rarely achieve. Given that the vast majority of Tanzanian students do not have access to any textbooks at all, the importance of interactive learning techniques such as those provided in this section must not be underestimated.

Many of these games and activities reinforce very basic mathematical concepts, so they are a great way to help solidify students' understanding of such crucial topics through a variety of approaches. Given some creative dimension to their learning, students become more enthusiastic about practicing their math skills.

\section{Games} \label{games}

	\subsection{Jeopardy} \label{jeopardy}
	\textbf{Applicable Topics:} Any topic as a review game\\	
	\textbf{Number of Players:} 3-6 Groups\\
	
	\noindent \textbf{How to Play:}\\
	Divide the class into equal groups and allow them to select their own team name. Designate one student from each team as the writer. Select 5 categories (see suggestions below) and write them at the top of the board on the far left. Beneath each category, write 100, 200 and 300, using colored chalk if possible. Each question's point value is based on its difficulty (100 - easy, 200 - average, 300 - difficult). Write a column for each team on the right side of the board and leave the center of the board empty.
	
	The first team selects a question by stating a category and a point value. The teacher writes the problem in the center of the board for all groups to see. Groups work together to solve the problem, but only the designated writer may present the answer. When an answer is found, writers must hold their notebooks high above their heads so that no further changes can be made, and everyone in the group must remain silent. The first group to finish with a correct answer gets the points and may select the next question. However, groups only get \textbf{one} attempt at each question, so if they are wrong, they are finished until the next question. If none of the groups answer correctly, the teacher gets the points! Once a correct answer has been given, have one student come up to the board to show the rest of the class. 
	
	This is a wonderful review game to play before an exam, since it also shows students what topics they need to study.\\
	
	\noindent \textbf{Extensions:}
	
	\noindent JEOPARDY works great as a review game for any topics covered in class. However, it can also be useful for teaching logical thinking skills, and may be incorporated as part of a Math Day-type promotional school event, or a Math and Science Conference or Competition. Here are some suggested categories for such game variants.
	\begin{itemize}
	\item \emph{Pattern Problems} - pattern recognition sequences for progressions of shapes, rotating figures, letters, etc. Give the first 3, students must find the following 2. (e.g. Abcd, eFgh, ijKl, $\_\_\_$, $\_\_\_$)
	\item \emph{Sequence Solvers} - Numerical sequences or series with various rules. Give the first 4, students must find the following 2. (e.g. 1, $\frac{1}{22}$, 333, $\frac{1}{4,444}$, $\_\_\_$, $\_\_\_$)
	\item \emph{Creative Classes} - Write 9 items on the board, mixed together randomly. Groups must find common characteristics among the items to classify them into 3 groups of 3. (e.g. Oxygen, Milk, Water, Pen, Table, Carbon Dioxide, Oil, Air, Rice - \textit{Gases, Liquids, Solids})
	\item \emph{Word Work} - Give students a word that applies to math or science. Groups must make at least four 3-or-more-letter words by re-arranging the letters. (e.g. MATHEMATICS - math, mat, the, them, $\ldots$)
	\item \emph{Design Dilemmas} - Groups must solve design problems using given constraints. For example:
		\begin{itemize}
		\item You must design a rectangular building having a volume of 64 m$^3$. What is the smallest surface area you can use?
		\item You need to fence a farm having an area of 48 m$^2$. What is the smallest length of fence you can use to border the full area of the farm?
		\end{itemize}	 
	\end{itemize}
	
	\subsection{Memory Matching} \label{memory}
	\textbf{Applicable Topics:} \nameref{fractions}, \nameref{decspercents}, \nameref{congruence}, \nameref{similarity}, \nameref{areaper}, \nameref{algebra}\\	
	\textbf{Number of Players:} 2-6\\
	
	\noindent \textbf{How to Play:}\\
	Prepare roughly 10 pairs of cards, with each set being a different pair of congruent shapes. Mix all cards together and place them in a grid, face down. Players take turns turning over 2 cards at a time. If the pair is a match of congruent shapes, they get to keep them and try another pair of cards. If they are not a match, they must turn them face down again, in their same places, and it is the next player's turn to try. The game ends when all cards have been paired off. The player with the most pairs wins.\\
	
	\noindent \textbf{Extensions:}
	\begin{itemize}
	\item Use pairs of similar shapes instead of congruent shapes.
	\item Make pairs of equivalent fractions, decimals or percentages. All students work together to determine if a given pair are equivalent.
	\item Make pairs with one card being a math equation and its pair being the solution. This can be applied to nearly any topic.
	\item Turn the game into a Concentration game. On a big sheet of paper the size of the full grid of cards, write down a math problem, e.g. a pair of simultaneous equations or a problem on area or perimeter. Without students seeing this sheet, cover it with the face down card grid. As students remove successfully paired cards, the picture beneath is revealed, piece by piece. The first student to solve the problem underneath is the winner. They may need to make inferences or assumptions while parts of the board are still covered.
	\end{itemize}
		
	\subsection{Around the World} \label{aroundtheworld}	
	\textbf{Applicable Topics:} \nameref{numbers}, \nameref{fractions}, \nameref{decspercents}, \nameref{geometry}, \nameref{areaper}\\	
	\textbf{Number of Players:} Entire class\\
	
	\noindent \textbf{How to Play:}\\
	Use \nameref{flashcards} to drill basic math operations. The first 2 students stand up next to each other at their desks. The teacher holds up a card, e.g. 5 $\times$ 2. The first students to give the correct answer moves on, the other student sits down. The winner then faces the next student in order, who stands up and another card is drawn. When a player is defeated, he or she must take the seat of the student who beat them. See who can make it the farthest without being beaten. Can anyone make it around the entire classroom and back to their original seat? It is important for the rest of the class to be silent while two players are facing off.\\
	
	\noindent \textbf{Extensions:}
	\begin{itemize}
	\item Instead of operations, have students compete in converting fractions, decimals and percentages.
	\item Show students simple shapes with dimensions and have them state the area or perimeter.
	\item Show students various shapes and they must be first to identify the shape, e.g. pentagon, rhombus, isosceles triangle, etc.
	\end{itemize}
	
	\subsection{BINGO} \label{bingo}
	\textbf{Applicable Topics:} \nameref{numbers}, \nameref{fractions}, \nameref{decspercents}, \nameref{trig}\\	
	\textbf{Number of Players:} Entire class\\
	
	\noindent \textbf{How to Play:}\\
	Each student makes a 4 $\times$ 4 or 5 $\times$ 5 table in their notebook. Students then fill their game card with random numbers from a given range specified by the teacher, e.g. from 1-40, even numbers from 1-100, etc. The teacher then calls out numbers one at a time. If students have that number on their cards, they place a marker (small square of paper) to cover it up. When a player has a full row, column or diagonal covered, they must shout, ``Bingo!'' and the teacher checks to confirm all of the numbers were called. Players then clear their cards and a new game begins.\\
	
	\noindent \textbf{Extensions:}
	\begin{itemize}
	\item Instead of just calling the numbers, give students a short problem that they must solve first. For example, instead of reading ``5,'' write on the board the equation, ``4 $\times$ 3 - 7'' Use this method to review BODMAS.
	\item Drill conversions among fractions, decimals and percentages. Students fill their cards by writing one form (e.g. decimals), but the teacher reads numbers in another form (e.g. fractions). Be sure to tell students what range to select numbers from when filling up their cards.
	\item Apply to trigonometry. Instead of B-I-N-G-O, column headers are $\sin$, $\cos$ and $\tan$. Students fill their cards with various values under each trig function, and the teacher calls out a particular trig function and angle. For example, the teacher calls, ``$\sin 30^\circ$'' and students having a ``0.5'' under their $\sin$ column get to cover that space.
	\end{itemize}
	
	\subsection{Battleship} \label{battleship}
	\textbf{Applicable Topics:} \nameref{coordgeo}, \nameref{geometry}, \nameref{relations}, \nameref{functions}\\	
	\textbf{Number of Players:} 2\\
	
	\noindent \textbf{How to Play:}\\
	Each player makes 2 coordinate planes out of paper or card. Both the $x$ and $y$ axes should span from -5 to 5. Players arrange their papers such that one is standing vertically against a stack of books and the other is resting in front of it on a flat surface, hidden from the view of the other player.
	
	Players secretly place their ships on the flat coordinate plane in front of them. Ships are placed across several coordinate points, but may not be placed on a diagonal. For example, Player 1's battleship is placed horizontally across 4 points, from (-2,-3) to (2,-3). Each player has 5 ships of lengths 2, 3, 3, 4 and 5. Ships should be clearly marked using bottle caps, colored pen or some other indicator and cannot be moved once placed.
	
	Player 1 begins by calling out a coordinate point. If the point lies on one of Player 2's ships, he or she says, ``Hit,'' otherwise it is a ``Miss.'' Player 1 uses his or her vertical coordinate  plane to keep track of hits and misses on Player 2's ships. Players go back and forth until all of one player's ships are sunk.\\
	
	\noindent \textbf{Extensions:}
	\begin{itemize}
	\item Apply to geometry by using shapes instead of ships. Players must hit every corner in order to sink the shape.
	\item Apply to relations and functions. Instead of placing ships, students graph a line on their coordinate plane, such as $y = x + 3$. All points on the graph must be hit, and students must state the relation or function of their opponent in order to win. How many points must be hit in order to find the equation of the line?
	\end{itemize}
	
	\subsection{20 Questions} \label{20questions}
	\textbf{Applicable Topics:} \nameref{numbers}, \nameref{fractions}, \nameref{decspercents}\\	
	\textbf{Number of Players:} 2\\
	
	\noindent \textbf{How to Play:}\\
	Player 1 thinks of a number and gives a range of possibilities to Player 2, e.g. ``I'm thinking of a number between$\ldots$'' 0 and 100, -20 and -10, 1,000 and 2,000, etc. Player 2 must ask questions to try and guess the number. Player 1 may only answer ``yes'' and ``no.'' Player 2 must ask questions like:
	\begin{itemize}
	\item ``Is it larger than 50?''
	\item ``Is it smaller than 10?''
	\end{itemize}
	
	Keep count of how many questions it takes to guess the number. Every question counts as one point. Each player gets a few rounds to choose a number and a few rounds to be the guesser. The player with the lowest score wins.\\
	
	\noindent \textbf{Extensions:}
	\begin{itemize}
	\item Play variants of the game where students must choose a fraction, decimal or percentage.
	\end{itemize}
	
	\subsection{Guess Who?} \label{guesswho}
	\textbf{Applicable Topics:} \nameref{numbers}, \nameref{fractions}, \nameref{decspercents}\\	
	\textbf{Number of Players:} 2\\
	
	\noindent \textbf{How to Play:}\\
	Players begin with a 10 $\times$ 10 card on which are written the numbers 1-100. Each player secretly chooses a number and writes it on a small paper to keep in front of them during the game. Be sure the other player doesn't see your number!
	
	Players take turns asking each other questions regarding properties of the other player's number. Questions can only be answered by ``yes'' and ``no.'' Examples of questions may include:
	\begin{itemize}
	\item ``Is your number a prime number?''
	\item ``Is your number a square number?''
	\item ``Is your number an odd number?''
	\item ``Is your number a multiple of 3?''
	\item ``Is your number a factor of 10?''
	\end{itemize}
	
	Students cross of numbers with an X or cover up numbers with markers as they eliminate possibilities from each other's answers. Play until someone wins best of 5.\\
	
	\noindent \textbf{Extensions:}
	\begin{itemize}
	\item Make special game cards for a game on fractions, decimals or percentages, as long as both players' cards contain the same numbers.
	\end{itemize}
	
	\subsection{Toka (Number Line)} \label{toka}
	\textbf{Applicable Topics:} \nameref{numbers}\\	
	\textbf{Number of Players:} 2+\\
	
	\noindent \textbf{How to Play:}\\
	Create a set of \nameref{flashcards} with single numbers written on them, ranging from -10 to +10. Be sure to write +'s for positive numbers and -'s for negative numbers. You will also need a \nameref{numberline} with a range of around -7 to 7.
	
	Start by using only the cards for -5 to +5 and shuffle them. Each player writes his or her name on a small square of paper with tape on the back (post-it notes also work very well). Players post their papers on the 0 as a starting point. 
	
	Players take turns randomly drawing a card to see how many spaces they have to move in the positive or negative direction and move their paper accordingly. Have students exaggerate movements on the number line	to learn by the hopping method. If a student draws a card that causes them to go off the edge of the number line in either direction, they are out and other students yell, ``Toka!''
	
	After the game has continued for a while, add in the higher numbered cards to the draw pile to make things more exciting. The last player remaining on the number line is the winner.\\
	
	\noindent \textbf{Extensions:}
	\begin{itemize}
	\item Use a game board with bottle caps or other items as game pieces to play on a table.
	\item Substitute \nameref{dice}, \nameref{spinners} or other random number-generating tools to see how many spaces must be moved.
	\item Even Form IV students can have much difficulty adding and subtracting negative numbers. Watch over as students play to make sure they are counting correctly.
	\end{itemize}
	
	\subsection{Snap!} \label{snap}
	\textbf{Applicable Topics:} \nameref{congruence}, \nameref{similarity}, \nameref{fractions}, \nameref{decspercents}.\\
	\textbf{Number of Players:} 2-5\\
	
	\noindent \textbf{How to Play:}\\
		You will need to make a pack of at least 40 cards. On each card write a fraction, a decimal or a percentage. Make sure there are several cards which carry equivalent fractions, decimals or percentages.
		
		Shuffle the cards and deal them out, face down to the players. Place a baton-like item, such as a pen, ruler or spoon, at the center of the table so that it is within equal reach of all of the players. Players go in a circle, placing one of their cards face up in the middle. The first player to see that a card is equivalent to another card face up in the middle must grab the baton and then, if correct, wins all of the cards in the middle. If, however, a player grabs the baton and there are no equivalent numbers showing, he or she must give one card to each remaining player. Players are eliminated when they run out of cards, but may still ``grab in'' to get back in the game if they can identify equivalent numbers played by the remaining players. The game continues until one player has won all of the cards.\\
		
		\noindent \textbf{Extensions:}
		\begin{itemize}
		\item Apply to topics like \nameref{congruence} and \nameref{similarity} by using cards with shapes drawn on them and having players identify congruent or similar shapes.
		\item As an alternative to grabbing a baton, have students shout, ``Snap!'' when they identify matching cards to minimize injuries and chaos.
		\end{itemize}
	
	
	
	
	
\section{Classroom Activities}	\label{classacts}
	
	\subsection{Fractions, Decimals and Percentages} \label{classactsfracsdecs}
	 
		\subsubsection{\underline{Fruit Fractions}}
		Put 6 pieces of fruit on three tables such that 3 are on one, two on another, and one on the final table. Use the same fruit for all 6, such as bananas or oranges, and make sure each piece is roughly the same size.\\
		
		\noindent Line up 10 students  outside the room. Let them in one at a time. Each student must choose to sit at the table where they think they will get the most fruit.\\
		
		\noindent Before the students enter, discuss the following questions with the rest of the class:
		\begin{itemize}
		
		\item Where do you think they will want to sit?
		\item How much fruit will each student get?
		\item If students could change tables during the game, would they?
		\item Is it best to go first or last?						\end{itemize}
	When all 10 students are seated, ask students to do the following:
	\begin{itemize}
	\item Write down how much fruit each student gets. Write the amount as a fraction and as a decimal.
	\item Write down the largest amount of fruit any one student gets. Write this amount as a percentage of the total amount of fruit on the tables.
	\end{itemize}
	
	\noindent \textbf{Extensions}
		
	\noindent Repeat the activity with a different set of students to wait outside the room. Try with a different number of tables, a different number of pieces of fruit or a different number of students.
	
		\subsubsection{\underline{Target 100}}	
		\textbf{Concept:} \emph{Multiplying by a number between 0 and 1 makes numbers smaller. Dividing by a number between 0 and 1 makes numbers bigger.}\\
		
		\noindent \textbf{Activity}
		
		\noindent Player 1 chooses a number between 0 and 100. Player 2 has to multiply it by a number to try and get as close as possible to 100. Player 1 then takes the answer and multiplies this by a number to try and get closer to 100. Students take turns doing this, and the student who gets closest to 100 in 10 turns is the winner.\\
		
		\noindent \textbf{Extensions}
		
		\noindent Change up the rules and play with division or other numbers such as 0.001 or 1,500. Encourage students to check each other's work by awarding points to the opposing side for catching mistakes.
			
	\subsection{Geometry} \label{classactsgeo}
	
		\subsubsection{\underline{Estimating Angles}}
		\textbf{Concept:} \emph{Angle is a measure of turn. It is measured in degrees. Angles can be acute (less than 90$^\circ$), right (90$^\circ$), obtuse (greater than 90$^\circ$), or reflex (greater than 180$^\circ$).}\\
		
		\noindent \textbf{Activity A}
		
		\noindent Player 1 chooses an angle, e.g. 49$^\circ$. Player 2 has to show that angle without using a protractor. Player 1 measures the angle with a \nameref{protractor}. Player 2 is given points equal to the difference between the angle drawn and the intended one. For example, Player 2's angle is measured to be 39$^\circ$, so Player 2 scores 10 points (49$^\circ$ - 39$^\circ$). Students take turns naming and drawing angles. The winner is the player with the lowest score.\\
		
		\noindent \textbf{Activity B}
		
		\noindent Each player draws 15 angles on a blank sheet of paper. They swap papers and estimate the size of each angle. Then they measure the angles with a \nameref{protractor} and compare the estimate to the exact measurement of the angles. Points are scored based on the difference between the estimate and the actual size of each angle. The player with the lowest score wins.
		
		\subsubsection{\underline{Tessellation Investigation}}
		\textbf{Concept:} \emph{A tessellation is a repeating pattern of one shape in more than one direction without any gaps. A semi-regular tessellation is a repeating pattern of two shapes in more than one direction without any gaps.}
		
		\emph{A regular shape will tessellate if the interior angle is a factor of 360$^\circ$. Semi-regular tessellations work if the sum of a combination of the interior angles of the two shapes is 360$^\circ$.}\\
		
		\noindent \textbf{Activity}
		
		\noindent Give students a collection of regular polygons. Ask them to find out:
		\begin{itemize}
	\item Which polygons can be used on their own to cover a surface without any gaps?
	\item Which two polygons can be used together to cover the surface without any gaps?
	\item Explain why some shapes tessellate on their own and others tessellate with a second shape.
		\end{itemize}
	
	\subsection{Algebraic Operations} \label{classactsalgebra}
	
		\subsubsection{\underline{Inverse Operations}}
		\textbf{Concept:} \emph{Addition is the inverse of subtraction and subtraction is the inverse of addition. If you do an operation followed by its inverse, you arrive where you started (e.g.} 7 + 2 - 2 = 7).
		
		\emph{When you are dealing with more than one operation, you arrive where you started if you do the inverse of each operation in the opposite order. For example:}
		\begin{center}
		$7 + 2 = 9$\\
		$9 \times 3 = 27$
		\end{center}
		\emph{then to reverse: $27 \div 3 = 9$, $9 - 2 = 7$.}\\
		
		\noindent \textbf{Activity}
		
		\noindent Give students instructions such as:
		\begin{itemize}
		\item I am thinking of a number. I multiply it by 5 then subtract 7. The answer is 58. What number was I thinking of?
		\item I'm thinking of a number. I multiply it by 3. I then subtract 6. I then divide by 2 and then add 5. The answer is 23. What was my number.	
		\end{itemize}
		
		\noindent \textbf{Extensions}
		\begin{itemize}
		\item Have students discuss strategies for finding the original number.
		\item Get students to find inverses of equations in two variables by working backwards.
		\item Use more advanced operations such as exponents, radicals and logarithms.
		\end{itemize}
		
		\subsubsection{\underline{Simultaneous Equations}}
		Write an equation at the top of the board, e.g. $x + y = 10$. Divide the rest of the board into two columns. Ask each student to do the following:
		\begin{itemize}
		\item Think of one set of values for $x$ and $y$ which make the equation on the board true. Do not tell anyone these values.
		\item Make up another equation in $x$ and $y$ using your values.
		\item Invite students one by one to say the equations they have made up. If their equation works with the same values as the teacher's equation, write it in the left hand column. If it does not work, write it in the right hand column.		
	\end{itemize}
	
\noindent Ask students to:
	\begin{itemize}
	\item Work out the values of $x$ and $y$ for each set of equations.
	\item Discuss the methods they used to solve each set of simultaneous equations.
	\end{itemize}
Study the two lists of equations on the board. 
	\begin{itemize}
	\item Are any pairs the same? 
	\item Can any of the equations be obtained from one or two others?
	\end{itemize}
	
	\subsection{Ratios} \label{classactsratio}
	
		\subsubsection{\underline{Body Part Ratios}}
	\textbf{Concept:} \emph{Ratio is the comparison of two quantities or measurements. Ratios can be written as follows: a:b, age:height, 2:3. Ratios show how many times larger or smaller one thing is compared to another.	}\\
	
	\noindent \textbf{Activity}
	
	\noindent Make a list of body parts that can be measured with a piece of string, such as:
	\begin{itemize}
	\item circumference of the wrist
	\item circumference of the neck
	\item circumference of the base of the thumb
	\item circumference of the waist
	\item distance from shoulder to finger tip
	\item height
	\item circumference of the head
	\end{itemize}
	
	\noindent Cut a length of string the same length as each body part in the list. Find the ratio of things like:
	\begin{itemize}
	\item wrist:neck
	\item waist:height
	\end{itemize}

\noindent \textbf{Estensions}

\noindent Investigate other body ratios. Record your findings by using the thumb as a reference value of 1. Find other ratios such as:
\begin{itemize}
\item nose length:thumb length
\end{itemize}	

	\subsection{Coordinate Geometry and Transformations} \label{classactscoordgeo}			
		\subsubsection{\underline{Coordinate Quadrilaterals}}
		\textbf{Concept:} \emph{Coordinate pairs give the position of a point on a grid. The coordinate pair (2,3) describes a point with a horizontal displacement of 2 and a vertical displacement of 3 from the origin.}\\
		
		\noindent \textbf{Activity}
		
		\noindent Draw a large pair of axes on the ground or on a large piece of card on the ground. Label the $x$ and $y$ axes.\\
		
		\noindent Place 4 bottle caps on the grid as the vertices (corners) of a quadrilateral. Record the 4 coordinate pairs. Make other quadrilaterals and record their coordinate pairs.\\
		
		\noindent Sort the quadrilaterals into the following categories: square, rectangle, rhombus, parallelogram, kite, trapezium. In each category look for similarities between the sets of coordinate pairs.		
		
		\subsubsection{\underline{Bottle Cap Transformations}}
		\textbf{Concept:} \emph{Transformations are about moving and changing shapes using a specified rule. Four ways of transforming shapes are reflection, rotation, enlargement and translation.}\\
		
		\noindent \textbf{Activity - Reflection}
		
		\noindent Every point has an image point at the same distance on the opposite side of the mirror line.
		\begin{itemize}
		
	\item Place 4 bottle caps, top-side up, to make a quadrilateral. Record the coordinate pairs. Place another 4 bottle caps, teeth-side up, to show the mirror image of the first quadrilateral reflected in the line $y = 0$. Record these coordinate pairs and compare to those of the original.	
	\item Show different quadrilaterals reflected in the line $y = 0$. Note the coordinates and investigate how the sets of coordinates are related.
	\item Make reflections of quadrilaterals in other lines such as $x = 0$ and $y = x$.	
		\end{itemize}
		
		\noindent \textbf{Activity - Rotation}
		
		\noindent All points move the same angle around the center of rotation.
		\begin{itemize}
		
	\item Place bottle caps, top-side up, to make a shape. Record the coordinates of the corners of the shape. Place another set of bottle caps, teeth-side up, to show the image of the shape when it has been rotated 90$^\circ$ clockwise about the origin. Record these new coordinates and compare to the originals.	
	\item Repeat for different shapes, recording the  coordinates before and after rotation and investigating how they are related.
	\item Try rotations of other angles like 180$^\circ$ clockwise or 90$^\circ$ anticlockwise.
		\end{itemize}		
		
		\noindent \textbf{Activity - Enlargement}
		
		\noindent A shape is enlarged by a scale factor which tells you how many times larger each line of the new shape must be.
		\begin{itemize}
		
	\item Place bottle caps, top-side up, to make a shape. Record the coordinates of the corners of the shape. Place another set of bottle caps, teeth-side up, to show the image of the shape when it has been enlarged by a factor of 2. Record the new coordinates and compare to the originals.
	\item Show different shapes enlarged by scale factors such as 5, $\frac{1}{2}$ and -2.	
		\end{itemize}
		
		\noindent \textbf{Activity - Translation}
		
		\noindent All points of a shape slide the same distance and direction.
		\begin{itemize}
		\item Place bottle caps, top-side up, to make a shape. Record the coordinates of its corners. Place another set of bottle caps, teeth-side up, to show the image of the shape when it has been translated left 4 units and record its new coordinates. Compare the two sets of coordinate pairs and investigate how they are related.
		\item Now try different translations and try to predict the coordinates of the image.
\end{itemize}			
	
	\subsection{Functions} \label{classactsfns}
	
		\subsubsection{\underline{Discover the Function}}
		Think of a simple function, e.g. $x \times 3$. Write a number on the left side of the chalkboard. This will be the IN number, though it is important not to tell students at this point. Opposite your number, write the OUT number. For example:
		\begin{center}
		$10 | 30$
\end{center}	
\noindent Show two more lines. Choose any numbers and apply the same function rule.
\begin{center}
$5 | 15$\\
$7 | 21$
\end{center}	
\noindent Now write an IN number only and invite a student to come to the board to write the OUT number.
\begin{center}
$11 | ?$
\end{center}
\noindent When students show that they know the rule, help them to find the algebraic rule. Write $x$ under the IN column and invite students to fill in the OUT column.
\begin{center}
$x | ?$
\end{center}
\noindent When students have shown that they know the function, move on to another.\\
\noindent\textbf{Tip:} This activity is best done in silence!\\

\noindent \textbf{Extensions:}
\begin{itemize}
\item Try functions with two operations, squares, cubes, radicals, etc.
\item Challenge students to find functions with two operations which produce the same table of IN and OUT values.
\item Challenge students to show that the function $x \times 2 + 2$ is the same as $(x +1) \times 2$.
\item How many other pairs of functions can they find that are the same?
\end{itemize}		
		
	\subsection{Sequences and Series} \label{classactsseqser}
	
		\subsubsection{\underline{Sequence Steppers}}
		\textbf{Concept:} \emph{A number sequence has a starting point and a step size. For example, starting at 3 and going up in 5's produces the sequence:}
		\begin{center}
		3, 8, 13, 23, $\ldots$
\end{center}

\noindent \emph{The Fibonacci sequence is made by starting with the digits 1 and 1. Each new term is made by adding together the previous two terms. That is,}
\begin{center}
1, 1, 2, 3, 5, 8, 13, 21, $\ldots$
\end{center}	

\noindent \textbf{Activity}

\noindent Have students make or imagine a number line stretching on both sides of them. Tell them to locate 0. Tell the students they are now going to go for walks along their number lines. Give them instructions such as:
\begin{itemize}
\item Start at 0, step on all multiples of 3. How many steps before you pass 50?
\item Start at 4 and go up in 7's. Will you land on 100?
\item Start at 5 and go down in 11's. How many steps before you pass -100?
\item Start at 9 and go up in a Fibonacci sequence. How many prime numbers do you land on before you get to 100? What are they?
\item Start at 7 and go up in 4's. As you land on each number, look at the units digit. When do they start repeating? How long is the cycle?
\item Start at -5 and go down in 3's. As you land on each number, look at the units digit. What is the pattern?
\item Start at 0 and walk along the line until you get to 10. Now fold your line around so that 11 ends up next to 9. Look at the other pairs you have created. What is 0 next to? What do you notice about these pairs of numbers?
\end{itemize}		
	
	\subsection{Volume and Surface Area} \label{classacts3d}
	
		\subsubsection{\underline{3-D Design Challenge}}
		\textbf{Concept:} \emph{Volume is the amount of space a solid takes up. It may be found by counting cubes or by calculations for regular solids.}
		
		\emph{Surface Area is the area of the net of a solid. It can be found by counting squares or by calculation for regular shapes.}\\
		
		\noindent \textbf{Activity}
		
		\noindent Students must construct 3-D shapes given certain design constraints. For example:
		\begin{itemize}
		
	\item You may only use 1 sheet of paper. What is the largest volume cuboid you can make?
	\item You need to make a box which has a volume of 96 cm$^3$. The box can be any shape. What is the smallest amount of card you need?
	\item You have a 24 cm $\times$ 24 cm square of card. You can make a box by cutting squares out of the corners and folding the sides up. Make a box with the largest volume. What is the length of the side of the cut-out squares? Try for other sizes of square cards or with rectangular cards.
	\item You have a rectangular piece of card which is 24 cm $\times$ 8 cm. What is the largest volume cylinder you can make?
	\item You are going to make a cylinder which must have a volume of 80 cm$^3$. What is the smallest amount of card you need?	
	\end{itemize}
	
	\subsection{Probability and Statistics} \label{classactsprob}
	
		\subsubsection{\underline{Probability Line}}
		\textbf{Concept:} \emph{Probability is the likelihood of an event happening. To describe the likelihood of an event happening, we use words like: very likely, unlikely, certainly, even, impossible, probable, very unlikely, no chance, definite, dead certain, etc.}\\
		
		\noindent \textbf{Activity}
		
		\noindent Tie a piece of string to make a straight line across the classroom. Peg cards with 0 and 1 written on them to either end of the line. This is a probability line that goes from 0 (impossible) to 1 (certain). Using clothes pins, peg cards on the line to show the likelihood of different future events. Make up events of your own and put them on cards on the line. Examples of events include:
		\begin{itemize}
		\item It will rain tomorrow.
		\item I will go to school tomorrow.
		\item I will throw a 6 on the die.
		\end{itemize}	
					
		\noindent \textbf{Extensions}

\begin{itemize}
\item Discuss where different word descriptions should be placed on the probability line: even, very likely, good chance, dead certain, possible, unlikely, no chance, etc.
\item Have students write events corresponding to the aforementioned descriptions. (e.g. for ``dead certain,'' Mr. Jack will eat ugali next week.)
\item Mutually exclusive events - Have students compare the probabilities of mutually exclusive events in terms of where they lie on the clothesline and how to describe them using words. (e.g. What is the probability that Kikwete will be re-elected? What is the probability that he will not be re-elected?)
\item Dependent events - Give students pairs of dependent events A and B. Where is B located on the clothesline given that A has happened? Where is B located given that A has not happened? (e.g. If you pass math class, what is the probability that you'll pass the NECTA exam? Alternatively, if you fail, what is the probability that you'll pass the NECTA exam?)
\end{itemize}
		
		\subsubsection{\underline{Left and Right}}
	Make a number line or board with 0 at the center and 5 evenly spaced numbers on either side. A player sits at each end. Use a bottle cap as a counter and start it at 0.
	\begin{itemize}
	\item Player 1 rolls 2 dice. Find the difference between the two numbers showing.
	\item If the difference is 0, 1 or 2, move the counter one space to the left.
	\item If the difference is 3, 4 or 5, move one space to the right.
	\item Take turns rolling the dice, calculating the difference and moving the counter. Keep a tally of how many times the right player wins and how many times the left player wins.
	\item Collect the results of all the games in the class. What do you notice about the results?
	\item Is the game fair? Why or why not?
	\item Challenge students to redesign the game so that the chances of winning are:
	\begin{itemize}
	\item better than losing
	\item worse than losing
	\item the same as losing
	\end{itemize}
\end{itemize}	
	
		\subsubsection{\underline{Feely Bag}}
	Put different colored beads (or bottle caps) in a bag, e.g. 5 red, 3 black and 1 yellow bead. Invite one student to take out a bead. The student should show the bead to the class and they should note its color. The student then puts the bead back in the bag. Repeat this many times. Stop when students can say with confidence how many beads of each color are in the bag.
	
		\subsubsection{\underline{Curious Combinations}}
	\textbf{Concept:} \emph{All possible outcomes can be listed and counted in a systematic way.}\\
	
	\noindent \textbf{Activity}
	
	\noindent How many ways can you arrange three different bottle caps in a line? Investigate for different numbers of bottle caps. Can you develop a relation between the number of items and the number of possible outcomes?





\section{Stimulating Puzzles}
%find the answer based on given numbers, get more ex's from teacher
	\subsection{SUDOKU} \label{sudoku}
	The classic Sunday paper SUDOKU puzzles are a wonderful tool for teaching logical thinking and problem-solving strategies to students. Each row, column, diagonal, and mini box must contain the numbers 1-9. \\
	
	\noindent Do an easy example with students, and then give them more to do on their own. Stress the fact that there is only \textbf{one} possible number for each box - just because a number \emph{could} fit does not necessarily mean that it \emph{does} fit. Teach concepts like process of elimination to remove possibilities. Even if kids do not like math, these puzzles will help to make them better thinkers!
	
	\subsection{Magic Squares}
	A magic square is a 3 x 3 square of numbers in which every row, column and diagonal add up to the same total or ``magic number.'' Here is an example of a Magic Square with 24 as its magic number:
	\begin{center}
	\begin{tabular}{|c|c|c|} \hline
	11 & 3 & 10 \\ \hline
	7 & 8 & 9 \\ \hline
	6 & 13 & 5 \\ \hline
	\end{tabular}
	\end{center}
	Create your own magic square using the numbers 1-9 and a magic number of 15. How many ways can you do it?\\
	
\noindent There are 880 different solutions to a 4 x 4 magic square using the numbers 1-16. How many of them can you find where the magic number is 34?\\
	
\noindent Find the magic number in these Magic Squares and then complete them:
	\begin{center}
	\begin{tabular}{|c|c|c| cccc |c|c|c|} \cline{1-3} \cline{8-10}
	6 &  &    & & & & &   &  & 10\\ \cline{1-3} \cline{8-10}
	7 & 5 & 3   & & & & &   & 7 & \\ \cline{1-3} \cline{8-10}
	 &  &    & & & & &  4 & & 5 \\ \cline{1-3} \cline{8-10} 
	\end{tabular}
	\end{center}
	Now try these, a little harder, where numbers are given but the reasoning is not so straightforward:
	\begin{center}
	\begin{tabular}{|c|c|c| cccc |c|c|c|} \cline{1-3} \cline{8-10}
	14 & 3 &    & & & & &  11 & 1 & \\ \cline{1-3} \cline{8-10}
	 &  & 13   & & & & &  9 &  & 7\\ \cline{1-3} \cline{8-10}
	8 & 15 &    & & & & &   & 15 & 5 \\ \cline{1-3} \cline{8-10} 
	\end{tabular}
	\end{center}
	
	
	
	\textbf{Solutions:}
	\begin{center}
	\begin{tabular}{|c|c|c| cccc |c|c|c|} \cline{1-3} \cline{8-10}
	6 & \textcolor{red}{1} & \textcolor{red}{8}   & & & & & \textcolor{red}{9}  & \textcolor{red}{2} & 10\\ \cline{1-3} \cline{8-10}
	7 & 5 & 3   & & & & &  \textcolor{red}{8} & 7 &\textcolor{red}{6} \\ \cline{1-3} \cline{8-10}
	 \textcolor{red}{2}& \textcolor{red}{9} & \textcolor{red}{4}   & & & & &  4 & \textcolor{red}{12} & 5 \\ \cline{1-3} \cline{8-10} 
	\end{tabular}
	\end{center}
	
	\begin{center}
	\begin{tabular}{|c|c|c| cccc |c|c|c|} \cline{1-3} \cline{8-10}
	14 & 3 & \textcolor{red}{10}   & & & & &  11 & 1 & \textcolor{red}{12}\\ \cline{1-3} \cline{8-10}
	\textcolor{red}{5} & \textcolor{red}{9} & 13   & & & & &  9 & \textcolor{red}{8} & 7\\ \cline{1-3} \cline{8-10}
	8 & 15 & \textcolor{red}{4}   & & & & &  \textcolor{red}{4} & 15 & 5 \\ \cline{1-3} \cline{8-10} 
	\end{tabular}
	\end{center}
	
	\subsection{24 Squares} \label{24squares}
	Surround a box by placing a digit on each side. Students must use all four numbers, together with any mathematical symbols they know, to produce an answer of 24. For example:
	\begin{center}
	\begin{tabular}{c c c  c c c c  c c c  c c c c  c c c} 
	& 3 &   &&&    & 5 &    &&&    & 8 &\\ \cline{2-2} \cline{7-7} \cline{12-12}
	4 & \multicolumn{1}{|c|}{} & 5    &&&    5 & \multicolumn{1}{|c|}{} & 5    &&&    1 & \multicolumn{1}{|c|}{} & 9\\ \cline{2-2} \cline{7-7} \cline{12-12}
	& 1 &	&&&   & 5 &    &&&    & 3 &
	\end{tabular}
	\end{center}
	Use this game to teach BODMAS as order of operations must be known to solve many of these puzzles. Arrange students in groups and see who can finish 10 the fastest.\\
	
	\textbf{Possible Solutions:}
	\begin{center}
	$5 \times 4 + 3 + 1 = 24 $\\
	$ 5 \times 5 - (5 \div 5) = 24 $\\
	$ 8 \times (9 \div 3) \times 1 = 24 $ 
	\end{center}
	
%open-ended, how many can you find/make?	
	\subsection{Four 4's} \label{fourfours}
	This easy to learn activity can provide many hours of interest and fun in your students. The idea is to express numbers from 1 to 100 using exactly four 4s and whatever mathematical symbols one knows. For example:
	\begin{center}
	$15 = 44 \div 4 + 4$
	\end{center}
or,
	\begin{center}
	$15 = (4 \times 4) - 4 \div 4$
	\end{center}
	There are 15 ways to create 9. Teach BODMAS since order of operations is key to the answers. It also drills fractions and other mathematical concepts. Can anyone make it all the way to 100?
	
	\subsection{Make a Century}
	By putting arithmetical signs in suitable places between the digits, make the following statement correct:
	\begin{center}
	1 2 3 4 5 6 7 8 9 = 100\\
	\end{center}		
	There is more than one solution. See how many you can find.\\
	\textbf{Useful Tip:} \emph{Here are a few solutions:}
	\begin{center}
	$123 - 4 - 5 - 6 - 7 + 8 - 9 = 100$\\
	$123 - 45 - 67 + 89 = 100$\\
	$(1 \times 2 \times 3) - (4 \times 5) + (6 \times 7) + (8 \times 9) = 100$\\
	$(1 \times (2 + 3) \times 4 \times 5) + 6 - 7 - 8 + 9 = 100$
	\end{center}
	\subsection{Intriguing Multiplication}
	Playing with his calculator one day, Jonny multiplies together the numbers 159 and 18 and obtains 7632. Upon reflection, he realizes that the equation
	\begin{center}
	$159 \times 48 = 7632$\\
	\end{center}
	contains each of the digits from 1 to 9, only once each. He can hardly believe his luck and feels the results must be unique. But he is wrong! There are several other pairs of numbers whose product and result are such that all the digits are used only once. Can you find any of them?\\
	\textbf{Useful Tip:} \emph{Here are some more examples:}
	\begin{center}
	$138 \times 42 = 5796$\\
	$198 \times 27 = 5346$\\
	$186 \times 39 = 7254$\\
	$157 \times 28 = 4396$\\
	$1963 \times 4 = 7852$
	\end{center}
	
	\subsection{Fraction Factory}
	The Babylonians had no notation for fractions such as $\frac{2}{3}$ or $\frac{3}{5}$, but only for unit fractions (fractions with 1 on the top, such as $\frac{1}{2}$ or $\frac{1}{5}$). This meant that a fraction like $\frac{2}{3}$ would have to be expressed as a sum or difference of unit fractions:
	\begin{center}
	$\cfrac{2}{3} = \cfrac{1}{3} + \cfrac{1}{3}$
	\end{center}
	Can you find ways of expressing fractions as the sum of difference of different unit fractions?\\
	Here are some examples:
	\begin{center}
	$\cfrac{1}{12} = \cfrac{1}{3} - \cfrac{1}{4}$\linebreak\\
	
	$\cfrac{2}{5} = \cfrac{1}{5} + \cfrac{1}{6} + \cfrac{1}{30}$\linebreak\\
	
	$\cfrac{3}{4} = \cfrac{1}{4} + \cfrac{1}{5} + \cfrac{1}{6} + \cfrac{1}{20} + \cfrac{1}{24} + \cfrac{1}{30} + \cfrac{1}{120}$
	\end{center}
	\textbf{Useful Tip:} \emph{One approach is to just add and subtract different unit fractions to see what results. But to make real progress, certain patterns need to be recognized. For example:}
	\begin{center}
	$\cfrac{1}{12} = \cfrac{1}{3} - \cfrac{1}{4}$\\
	\end{center}
	is a special case of
	\begin{center}
	$\cfrac{1}{n(n+1)} = \cfrac{1}{n} - \cfrac{1}{n+1}$
	\end{center}
	
	
\section{Brain Teasers and Mind Benders}
%how did they do it? Find the one correct answer.
	\subsection{Hundreds, Tens and Units}
	Take any three digit number such as 235. Write down the number formed by putting its digits in reverse order: 532. Subtract the smaller number from the larger number.
	\begin{center}
	$532 - 235 = 297$\\
	\end{center}
	Now write down the number formed by reversing the order of the digits in the answer and add to the answer itself.
	\begin{center}
	$297 + 792 = 1089$\\
	\end{center}
	When you have tried this for a few more numbers you will be able to predict the number and baffle friends.
	
	\textbf{Answer:} \emph{The answer is always 1089 unless the first number chosen has equal hundreds and ones digits, in which case the first subtraction would yield zero.}
	\subsection{Roll a Penny (or 100 /= Coin)}
	Penny A is rolled around Penny B without slipping until it returns to its starting point. How many revolutions does Penny A make?
	
	\textbf{Answer:} \emph{A makes 2 revolutions. Starting on the left side of Penny B, A will be upside-down when it has rolled to the top of B, upright when it is on the right side of B, upside-down again when at the bottom of B, and upright again when back at the start.}
	
	\subsection{Invert the Triangle}
	A triangle of bottle caps is made with one cap on the top row, 2 on the second row, 3 on the third and 4 on bottom. What is the smallest number of caps that must be moved to turn the triangle upside-down?
	
	\textbf{Answer:} \emph{Three caps need to be moved. Move the three corners: the top one to the right of the row below it, the bottom-left one up so now there are 4 caps on top, and the bottom-right moves to the very bottom. The row of three was left unchanged, and the original row of 4 now has 2 caps.}
	\subsection{Dividing Lines}
	Investigate the greatest number of regions you can make on a sheet of paper using a given number of lines. Create a table to track your results:
	\begin{center}
	\begin{tabular}{|c|c|c|c|c|c|c|c|} \hline
	Number of Lines & 1 & 2 & 3 & 4 & 5 & 6 & 7\\ \hline
	Number of Regions & & & & & & & \\ \hline
	\end{tabular}
	\end{center}
	Have students complete the table above and compare their results. Who was able to make the most regions? (\textit{Hint: One can make 7 regions with 3 lines.}
	
	\textbf{Answer:}
	\begin{center}
	\begin{tabular}{|c|c|c|c|c|c|c|c|} \hline
	\textbf{Number of Lines} & \textbf{1} & \textbf{2} & \textbf{3} & \textbf{4} & \textbf{5} & \textbf{6} & \textbf{7}\\ \hline
	\textbf{Number of Regions} & \textbf{2} & \textbf{4} & \textbf{7} & \textbf{11} & \textbf{16} & \textbf{22} & \textbf{29} \\ \hline
	\end{tabular}
	\end{center}	
	
	
	\subsection{The Ingenious Milkman}
	A milkman has only a 5 liter jug and a 3 liter jug to measure out milk for his customers. How can he measure 1 liter without wasting any milk?
	
	\textbf{Answer:} \emph{First fill the 3 liter jug. Next pour the 3 liters from this jug into the 5 liter jug. Again fill the 3 liter jug and then pour from it into the partially filled 5 liter jug until it is full. This leaves exactly 1 liter in the 3 liter jug. It would be possible to measure any whole number of liters by measuring single liters in this way. Clearly there are more efficient ways of measuring most quantities (5 and 3 can be done directly, 6 = 3 + 3, 8 = 5 + 3, etc. How about 7 liters or 4 liters?}
		
	\subsection{A Weighing Problem}
	A grocer has a pair of scale pans and 4 weights. The weights are such that with them she can correctly weigh any whole number of kilograms from 1 to 40. How heavy is each weight and how can he manage to weigh all of the different weights?
	
	\textbf{Answer:} \emph{The weights are 1 kg, 3 kg, 9 kg and 27 kg. By putting the weights on either scale pan, all of the weights from 1 to 40 can then be achieved. For example:}
	\begin{center}
	$11 = 9 + 3 - 1$\\
	
	$20 = 27 + 3 - 9 - 1$
	\end{center}
	
	\subsection{A Question of Balance}
	In a box there are 27 new red ping pong balls, all looking exactly alike. However, it is known that one of them is faulty and weighs more than the others. Given that you have a balance but no weights, show how, by comparing sets of balls against each other, you can find the faulty ball in only 3 balances.
	
	\textbf{Answer:} \emph{Compare 9 balls with 9 balls and leave 9 in the box. If the scales balance then the heavy ball is in the box, but if not, then the 9 balls which weigh down the scale contain the heavy ball. Either way, the faulty ball has been narrowed down to a set of 9 after the first balance. Divide this set of 9 into three sets 3. After this second balance you will have narrowed the faulty ball down to a set of 3, and one more balance identifies the faulty ball.}
	
	\emph{A similar but much harder problem is to find the odd ball from a set of 13 in only three balances.}