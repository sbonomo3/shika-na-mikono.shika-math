\chapter{Challenges Facing Math Teachers and Strategies for Success}

\section{The Actual Teaching Part}
During a volunteer's service, he or she may encounter a variety of challenges specific to teaching mathematics in Tanzania. Of course, the strategies for meeting these challenges depends entirely on the individual, but given here are some possible approaches based on the experiences of previous volunteers.

\subsection{Language Barriers}
All Tanzanian secondary schools are supposed to be taught in an English medium. Nonetheless, almost all students are learning English as a second or third language, and therefore struggle to understand an English-speaking volunteer. This frustration can lead students to quickly give up, especially in a subject that they already deem to be difficult such as mathematics.\\

\textbf{Strategies:}
\begin{itemize}
\item\emph{Keep at it!}: It may seem that, even after a long time, students are not catching on to English and don't understand a word of what you are saying. Repeated poor test scores may encourage you to give up on using English and resort to teaching in Kiswahili. Depending on the situation, this may be the best option, but before dropping English with your students, remember that learning is a very long process. The only way for them to improve their English skills is to hear and speak it everyday, and your class may be the only opportunity for them to practice. Simply showing your students that you are not willing to give up on them but rather continue to challenge them to learn English can be an inspiration and possibly motivate them to follow your example and not give up on themselves. Besides, you may find that whether you speak in English or Kiswahili, the students who want to learn will continue to make an effort, while those who are unmotivated will not become extremely studious just because you are speaking their language.

\item\emph{Show, don't tell}: Mathematics can be thought of as a universal language of its own. When it comes to teaching math (especially to Form I students), observation and practice can be much more beneficial tools for learning than lecturing. However, many students have developed a tendency to learn by rote memorization, so it is very important to use as many examples as possible in class to illustrate all of the distinctions in a new topic or principle.

\item\emph{Use student helpers}: Having well-performing students assist you in explaining a concept in Kiswahili can help others to understand more clearly, although if used too much, this strategy can lead students to stop listening to you altogether if they know a Kiswahili explanation is on the way.

\item\emph{Group work}: Getting students to work in groups gives them a forum to discuss problems and ask each other questions in their own language. As a teacher, you need to be present during this time to offer additional help and to make sure the students stay on task.
\end{itemize}

\subsection{Slow learners / Unmotivated students}
You will almost certainly find that your students have a wide range of math abilities, and that there are many who struggle to move at a pace you would like. This may be due to several factors: they never had a math teacher in primary school, they have problems at home interfering with their studies, they are just lazy, etc. Regardless of the reason, you will need to decide how to manage a class full of students moving at very different speeds.\\

\textbf{Strategies:}
\begin{itemize}
\item\emph{Practice, Practice, Practice}: The content of mathematical concepts builds upon itself. First a child needs an awareness of numbers (e.g. the child has 2 brothers). Then a student learns operations on numbers, which builds off of number awareness, and so on. The student uses what he or she knows to understand new mathematical concepts.

Therefore, it is so important to reinforce what is being taught. The student needs to practice all types of problems related to new knowledge learned. The teacher must drill, drill, drill: short quizzes or quick games are very helpful to ensure the progression of learning math skills.

As the math student progresses, he or she will be able to do more complex problems using many concepts taught in previous math classes. The student will be successful and enjoy math in school and in daily life.

\item\emph{REVIEW!}: Every class period should begin with a review from the previous class. This helps to reinforce every topic, allows students who were absent a chance to see what they missed, and provides a great opportunity to engage the students at the start of class with a board race or review game to get them excited for the day's math lesson.

\item\emph{Stream by ability}: If your school uses multiple streams, you may want to suggest streaming each form by student ability. That way, you can move more quickly with the fast learners and spend more time with the slow learners to accommodate each student's ability in math.

\item\emph{Focus on motivated students}: If you have a large number of students or classes to teach, you may find it in your best interest to focus your attention on the students who are very motivated to learn mathematics. You may find that many students are not willing to make the effort required to improve their own abilities. It's true that you are their teacher and may want to help every student, but at the same time you cannot force them to put forth an effort, and even if you could, doing so would not help them to take responsibility for their own education. You only have so much time to dedicate towards such a large number of students, so you must decide how to best utilize the time you have. 
\end{itemize}

\subsection{Ongoing assessment}
It can be difficult to monitor student progress with such large class sizes and limited class time. However, receiving continuous feedback from your students is necessary to your success as a teacher.\\

\textbf{Strategies:}
\begin{itemize}
\item\emph{Homework}: Homework may or may not be a feasible option at your school depending on what the students' after-school schedules are like. They may have a large number of domestic responsibilities after classes that eat up their time, but if not, homework assignments can be a great way to give students additional practice and can help you to identify how well the class has understood a concept. Marking hundreds of notebooks can be a daunting task for the teacher as well, so be sure to take your own time into account when assigning homework. If possible, try to give students at least 2 problems to work on by themselves after each new topic: one simple question to check for understanding, and one question to challenge them on something you didn't explicitly go over in class.

\item\emph{Pre- and post-tests}: For each chapter, give the students a pre-test to see if they already know a certain concept and you don't have to go over it in as much detail in class. Then at the end of the chapter give a post-test to see how much they improved from before your teaching. Show the results side by side to the students so they can see the progress they are making! A great motivation for the students and for you as a teacher!

\item\emph{Review games}: Games can be a wonderful way to engage students while getting useful feedback on what they have learned. Competitive games (especially boys vs. girls) will ensure that the students are trying their hardest to solve problems, so you can test them with questions of varying difficulty to see where they need improvement as a class.\\
For more ideas, see \nameref{cha:gamesandactivities}.

\item\emph{In-class assessment}: The size of your classes may dictate whether or not this is feasible on a regular basis, but assigning problems and checking answers in class is helpful in that you have immediate feedback on a topic that is fresh in the students' minds. You also are able to watch the students to see if they are copying off one another, and you may get an idea for who the fast learners are by the students who consistently are the first to finish.

\item\emph{Pop quizzes}: This can be helpful if you are concerned that only a small number of your students are paying attention and doing their work. Students will be more likely to pay attention every day if they know there is the possibility of having a pop quiz on the topic at any time. Nothing like striking fear into the hearts of children to motivate math learning!
\end{itemize}

\subsection{Class size / management}
Tanzanian classes can typically run upwards of 60-80 students, if not more. Regardless of your teaching background, you will likely need to develop new strategies for managing such a large class in order to utilize your time effectively. Large class sizes can make it difficult to accomplish certain tasks like grading homework and in-class assignments, keeping track of attendance, and making sure every student is taking notes during class.\\

\textbf{Strategies:}
\begin{itemize}
\item\emph{Group work}: There are many thinking styles of different students. Using group work helps to assist various patterns of learning, and so large classes can actually facilitate more learning if group work is used. The students need to be taught to work together as a team, though each student should write to facilitate individual learning. Math can not be discussion only, but the step by step thought process must be written by each student. The students should work as a team to accomplish their math calculations. When a group is finished, the answer is written on board and explained by a student. Questions are asked and the problem is discussed by the class.

As the teacher moves around the large classroom, he or she is reinforcing good behavior. Also the teacher can mark or write a motivating comment in the exercise book. Student love this! If a group needs help, frequently a neighboring student can be asked to teach the group who needs help. This technique can be very helpful in a large classroom.

\item\emph{Utilize class monitors}: Each class should have a monitor and a monitress, who should be responsible for keeping track of attendance in class everyday. You can use these students to help you track who is or is not coming to class on a regular basis, and also to bring student notebooks to and from your office for grading assignments.
\end{itemize}

\subsection{Teaching methodology}
Every teacher has different preferred teaching methods. Yours will likely result from trial and error and seeing what works best for your students.\\

\textbf{Strategies:}
\begin{itemize}
\item\emph{Discovery}: One of the best mathematics teaching techniques is to teach the students to discover formulas that are being learned by guided questions. For example, they can work many examples by picture to find perimeter of a rectangle. After adding length, width, length and width many times to find perimeter, the students can be guided by questions to discover $2l + 2w = P$. In this method of math teaching, the student will understand and remember the formula rather than memorizing. All formulas can be discovered because the formula does work, so it must have meaning. If the teacher is consistent in teaching through discovery, the teacher will be delighted with questions and discovery of his or her students. Learning is happening!! Teaching is successful!! 

Math learning is increased when senses are used. To use Math Tools, the students are using sight and
touch. To use a cylinder 3-D object, the students can discover the formula for surface area. This area is
found by taking the area of a rectangle around the cylinder and the areas of 2 circles (top and bottom). Again, this is a great method to teach and to learn!!

\item\emph{Balancing theory and examples}: Math is a subject learned by doing practice, but don't forget to include some of the theory and reasoning behind the math, or students will turn to simply memorizing formulas without realizing how to properly apply what they have learned. This is where review becomes helpful also, so that students can internalize which concepts are related to which topics.

You can also teach students about the history of math by, for example, briefly showcasing a Scientist of the Week. Because there is such a variety of thinking styles among students, perhaps teaching it from a historical point of view will really appeal to some students who otherwise would have been completely disinterested in math. Give some information about the lives of Pythagoras, Newton, Descartes, or other scientists and mathematicians that they may hear about in their theorems and principles. Show students that these are real people who made great contributions to our current understanding of math, and now we need new mathematicians to continue their work. Allow students to do a group project to research a scientist using Encarta or other textbooks and give a short presentation to the class. They will internalize this information because they took ownership of the project and did their own independent learning.

\item\emph{I do, we do, you do}: A math teacher must show students how to properly solve problems, but also give them the freedom to take concepts to the next level on their own. A simple example is done as a demonstration for the class, followed by a slightly more complex problem done together, and finally students work on their own to solve problems of varying complexity. This hierarchy, if kept consistent, can develop a sense of trust between teacher and student and can instill a logical progressive mental approach for students.
\end{itemize}

\subsection{Teaching towards the test}
You may find that there simply isn't enough time in the school year to teach every topic in the syllabus with the time required for student understanding. In addition, from analyzing past exams, you may see that certain topics rarely if ever make their way onto a NECTA exam, and that certain topics are more heavily weighted than others. So, do you skip the less-used topics to save time for the topics that they will actually be tested on, or continue to cover each topic to develop a more thorough math education?\\

\textbf{Strategies:}
\begin{itemize}
\item\emph{Skip the unused topics}: The students (namely Form III's and IV's) are aware of the fact that their entire 4-year education essentially boils down to a couple weeks of NECTA testing at the end of Form IV. They may even encourage you to move on to the topics that are more heavily weighted on the exams to help increase their performance. So you may decide it is in the students' best interest to properly prepare them for their exams by teaching towards the test.

\item\emph{Comprehensive learning}: Another option is to forget about the NECTA exam and focus on developing logical minds and a thorough understanding of mathematics within the students. Even if you were to teach strictly towards the exam, there are many factors affecting the results that are outside of your control, such as each student's preparation, the difficulty of the problems given on the exam, and the diligence of the graders in marking the exams (who are often paid by the number of tests they complete and may not have as much of a vested interest in your students' results as you do). So you may find it more beneficial to the students in the long term to try to focus on teaching them a problem-solving approach to use in life, rather than just making sure they get enough points on a test.
\end{itemize}


\subsection{USA/TZ different learning methods (GCF, LCM, etc.)}
Students learn certain topics in primary school differently than they are typically taught in America. They may have an instilled method of solving problems like finding GCF and LCM of a set of numbers, and if you are not careful, you may end up really confusing them by introducing a method they have never seen before.\\

\textbf{Strategies:}
\begin{itemize}
\item\emph{Ask the students}: Ask your students how they would solve a problem that they may have learned in primary school. Does it match the way you learned it or how the textbook explains it? Giving students a pre-test before each chapter can help you to identify Tanzanian methods of solving problems.

\item\emph{Utilize Tanzanian teachers}: Ask a fellow math or science teacher how they learned a topic when they were in school. If necessary, ask for their help in teaching a particular topic using a method that you are not familiar with.
\end{itemize}

\subsection{Prepping for NECTA exam}
Whether or not you decide to cater your teaching towards the NECTA exam, you can still help students prepare for the test early on. Whichever forms you are teaching, but especially Forms II and IV, which have NECTA exams at the end of the year, you can help the students get ready mentally for their tests throughout the school year so that they are more prepared and less worried come NECTA time.\\
For more on the format of the math NECTA exam, see \nameref{cha:nectaformat}.\\

\textbf{Strategies:}
\begin{itemize}
\item\emph{Use NECTA format when making exams}: All of your midterm and terminal exams can be written in the same format of NECTA in order to familiarize students with the layout, number of questions, and how to properly manage their time during the exam. Give them less time than they would normally get to challenge their speed as well! They may hate you for making such difficult exams at first, but better to be taken out of their comfort zone earlier rather than being shell-shocked on the actual NECTA exam.

\item\emph{Use old test papers for in-class problems}: Before starting a new topic, quickly browse some of the previous years' NECTA exams to see what kind of questions they are using for that topic. If you can get students comfortable with the wording and language of common exam problems while teaching the topic, then why not do it? NECTA exams often repeat problems or use very similar ones from year to year, so exposing students to a particular style of problems from a topic could be a great help to them later on.

\item\emph{Offer additional practice exams}: Use a previous year's NECTA exam or create your own to give to interested students as practice. Even the best students may be very comfortable with the math content, but also need to know how to work quickly and efficiently on the exam in order to score an A. Give them a Lightning Round of Section A questions, then Section B, then Section A again. Make it like a sports training drill. Don't just practice until they get it right, practice until they can't get it wrong! This is the level of comfort and confidence students must have in order to succeed on the NECTA exam. Confidence comes from experience!

\item\emph{Show students the math NECTA ``formula''}: The format of the math NECTA exam is very predictable! Section A topics carry only 3 or 6 points, whereas a single Section B topic (Accounts, Statistics, Linear Programming) can carry 10 marks. Students only have to choose 4 of the 6 questions from Section B, so they should decide which topics they are most comfortable with among these. 

Help students with their time management by breaking down how much time they should spend on each problem. For example, spend 10 minutes on each of the 10 problems from Section A (for 100 minutes), and 20 minutes on each of the 4 Section B problems (for 80 minutes) for a total of 180 minutes, or 3 hours. Then drill them with individual practice questions, giving only 8 minutes for Section A problems and only 15 minutes for Section B problems.
\end{itemize}

\subsection{What to do when everyone fails}
So now you've put in a ton of effort inside and outside of the classroom for a long time, given practice problems, graded homework, played games, done review, and guess what? They all still failed your exam. Now what? You may not want to show them the results in the fear that they will all become discouraged and hate math. Will you make future tests easier to try and build their confidence, or continue to turn on the heat to challenge them and show them what the NECTA exam will be like?\\

\textbf{Strategies:}
\begin{itemize}
\item\emph{Remain Positive!}: Even if your students did not perform as well as you would have liked on an exam, remember that your attitude towards math strongly influences those of your students. If you get frustrated and give up, how can you expect them to stay motivated? Search for more ways that you can help, what else can you do? Get feedback from the students - what is helpful for them and what isn't? Remember to always praise good performance rather than just punishing the bad, and let the students know when they have done something right. Tanzanian students do not often get positive encouragement at school or at home, so don't forget to give credit wherever it is deserved.

\item\emph{Focus on improvement}: Even if the entire class got a zero on your exam, that just means there is more room for improvement. Tell students to each try and increase their scores by 5 points for the next quiz. Start slow and take baby steps. Your successes as a teacher and their successes as students will not be seen overnight.

\item\emph{Go easy to build confidence}: Truly, you don't want to discourage students from studying mathematics, so it can be very helpful to start out slow, especially for struggling students, and give them time to develop confidence in a particular topic before moving on to the next. Also, as your teaching style will likely be very different from what they have been used to, it will probably take a while for students to become comfortable in the learning environment that you are creating. But at some point, you will still need to challenge them and take them out of their comfort zones, or their progress will become very limited.

\item\emph{Keep it challenging}: It can be greatly beneficial to hold students to high standards of performance and to make them aware of the level of difficulty present in the national exams and the effort required for success. However, this strategy may be best utilized in a gradual way. Giving students reasonable challenges in their math learning over time, and steadily increasing the level of difficulty of your exams may allow students to more easily adapt to your teaching methods and enjoy the challenges you present, rather than become discouraged by them. 
\end{itemize}



\section{Creating a Positive Math Culture}
In addition to the day-to-day teaching challenges faced by volunteer math teachers in Tanzania, it can also be difficult to change pre-conceived negative sentiments towards mathematics that have been instilled in students and teachers alike. Getting students to believe that they can succeed in mathematics will likely prove to be a much greater challenge than simply teaching them the material, and indeed this is a challenge that is felt across Tanzania and even in many developed parts of the world. Provided here are some common challenges and myths perpetuated by a negative math culture, along with some suggested strategies for combating them.

\subsection{Students don't like math / Math is hard / Ugonjwa wa taifa}
You may hear the term ``Ugonjwa wa taifa'' (National disease) being thrown around by fellow teachers and students regarding the state of mathematics in Tanzania. This pretty accurately sums up the general sentiment held by many Tanzanians that math is difficult and a lost cause for students to learn.

\emph{But this is not true!!!} There is no disease preventing students in this country from learning math, only a negative attitude and a laundry list of excuses! The best combatant towards a negative attitude is a positive attitude, and it can be just as contagious with the proper inspiration from even a single dedicated individual. Do not allow your students to buy into the idea that they are incapable of doing math. It is this culture of mathematics that must be changed first if any changes in academic performance are to be expected in Tanzania.\\

\textbf{Strategies:}
\begin{itemize}
\item\emph{Math Day}: Choose a Saturday to have Math Day at your school. Promote math, have a local guest speaker, (other teachers at your school) come to talk about the importance of studying math. Show some of the many daily life and career applications of math. Have a school-wide competition using math puzzles and games (see \nameref{cha:gamesandactivities}). Give out prizes and awards. Make math fun for students and show them that you and the rest of the school are not giving up on mathematics!

\item\emph{Mathletes\slash Hisaba-Team}: Start a Mathletes  club in your class or for the whole school. Divide students up into teams and give everyone a short 5-question quiz on a given Topic of the Week. The team with the highest average score wins! If you can keep the teams consistent, have weekly head-to-head match-ups among the teams, and keep a running standings board for the whole school to see. The students will enjoy the comparisons to football standings, and will love it even more if they can pick their own team names! Because it is a team effort and every score counts, students will encourage and help each other study for the coming week's topic.

\item\emph{Math practicals and projects}: Just because there is no practical component to the NECTA exam, that doesn't mean math isn't a practical subject. Have students build and use a \nameref{clinometer} to apply principles of trigonometry to find the height of a building. Assign a \nameref{statistics} project for the class to collect, organize and present data. Stress the use of the Scientific Method.

As an introduction to statistics, the students can be asked to collect data for the ages of 20 students in school. The students then will organize the data by tallying the results. Then a distribution table can be created with a histogram. This project is a great way to begin Statistics in Form 3. The components of data collection, organizing data in a table and graphing the results in a histogram are made clear in the project.

Projects can also assess the learning of a topic. Students can be asked to create 3-D objects and to demonstrate the formulas for surface area and volume. Students love to create these 3-D objects! This reinforcement helps the students to maintain formulas and processes to find surface area and volume for the NECTA examination.

\item\emph{GAMES!}: Using in-class review games such as \nameref{jeopardy} are a fantastic way to keep kids excited about math during class and can also provide great feedback leading up to a test.

\item\emph{Reward good performance}: Give students an incentive to come to class, do homework, ask for additional practice problems and study for tests by creating a school store and points reward system. For every extra problem they complete, they get one point to be used towards prizes like pens, pencils, erasers, math sets, four figure books, English dictionaries, etc. Get donations from friends and family back home to supply your store! You can even keep a leaderboard of the students with the most points in class to encourage others to catch up.

Or, if your school has a projector (or even your own laptop computer), offer tickets to Movie Nights on the weekends for well-performing students from that week. Every time a student does something positive, their name goes in the Movie jar. Every time they get in trouble, their name goes in the Kazi (Work) jar. At the end of the week randomly draw names from the jars to see who gets to watch Spiderman and who has to clean your toilet!

Recognize your bright students by announcing the top performers on each test. Give them stickers and write nice remarks on their papers. Allow them to receive recognition from all of their peers for their hard work. Give regular Student of the Month awards to the best boy and girl at the morning assembly so that the teachers can notice them as well. Once students start to see the positive benefits of studying math, more will follow in their footsteps. Success is contagious!

\item\emph{Math Conferences\slash Seminars}: Many volunteers get involved in regional Boys and Girls Conferences to educate students about HIV\slash AIDS, malaria, and other health issues. Organize a Math and Science Conference in your region to promote math! Bring your best students and give them a chance to meet and compete with the best students from nearby schools. They will return to your school and tell the other students all about how much fun it was and how they got to see the big city and meet students from other schools. The others will try to work harder so they can go next time.

It may be difficult to secure funding for this kind of a seminar, since the majority of grant funding tends to come from organizations supporting HIV\slash AIDS and malaria initiatives (PEPFAR is a major funding source for these conferences). So include these sessions in your conference and relate the topics to math and science! Have kids work on a project to organize and present the statistics on HIV\slash AIDS and malaria in their communities. What is the probability of a person living with either one? How many 3-D objects can you make out of a bed net to cover your bed at night? Which one uses the least surface area? What is the biology behind these illnesses? Students will learn that they can use math and science to help study and find solutions to these problems in their communities.

\end{itemize}

\subsection{Older students have given up on math (Form III, IV)}
You may encounter a good number of Form III and IV students who have already checked out of math because they are certain they will fail the math NECTA exam. They may reason that they never learned math throughout primary school or the first few years of secondary school, so why should they start now? It's too late, so might as well just quit now, right?

WRONG! Just because a student has struggled in the past does not mean they should give up. Sure, they were never successful learning math before, but they never had YOU as a math teacher before. You can still help these students to have success in mathematics and gain confidence in themselves.\\

\textbf{Strategies:}
\begin{itemize}
\item\emph{Extra Help}: Many students who give up on math do so because of some particular topic in their past learning that they got hung up on and did not understand very well, and after that they could never keep up with the new material and fell farther and farther behind. For many Tanzanian students, these can be topics such as counting, adding and subtracting. They have the capability to do it, but they've never been able to move at the pace of learning that they need to understand. Spending extra time with your slow learners and showing them that they are capable of doing mathematics when they put forth an effort can make a huge difference in their lives and give them a confidence and sense of accomplishment that they've never known before.

\item\emph{Non-math critical thinking and problem-solving}: So maybe a student doesn't enjoy doing math problems, but they can still be a logical and critical thinker! Get them to do puzzles and logic games (see \nameref{sudoku}) to improve their critical thinking and problem-solving skills.

\end{itemize}

\subsection{Math isn't useful / Don't use math in real life}
If you ask students why they don't like to study math, a common response you may get is that they never use math in their lives. Civics and History are practical, they may say, for learning about the government and social practices, but they will never use math, so why do they need to learn it? In order to give students the proper self-motivation to want to learn math, you must show them that they are very, very wrong about this assumption, and in fact math is used in nearly \emph{every} facet of their lives.\\

\textbf{Strategies:}
\begin{itemize}
\item\emph{World of science and technology}: We live in a world of science and technology. Even in Tanzania, nearly everyone has a cell phone, many people use computers and access the internet, and villagers are implementing solar power to provide electricity in rural areas. In order to keep up with the rapid pace at which the world is moving in terms of globalization and advancing technology, it is essential for all students to have at least a basic understanding of math and science.

\item\emph{Math in every subject}: Math is used in all other subjects: geography in statistics, civics in population, biology in genetics, physics in vectors, language in counting, chemistry to balance equations\ldots the list is limitless. Therefore, to succeed in other subjects, the student must understand mathematical concepts.

\item\emph{Everyday life applications}: Math is used during any money transaction: every time you go to the market or a store, you are using math. A business owner must be proficient in math to ensure she or he is making a profit. Even cooking requires knowledge of the proper ratios of items to mix. If you use these applications in your example problems, it will be easier for students to understand math concepts.

\item\emph{Math in many professions}: Math is used in many professions: Tailors use math to make clothes, carpenters use measurements to build furniture, engineers use math to plan roads, shop owners use math to operate his business
and to exchange goods for cash, school accountants use math to assess the school budget\ldots Every professional uses math in some area to be successful in his or her career.

\end{itemize}

\subsection{Girls can't do math}
A common myth in Tanzanian secondary schools is that girls cannot do math as well as boys. This attitude tends to reveal itself through test results, and if you are at a co-ed school, you may see a similar trend. One possible explanation is that when growing up, girls are often given a heavier work load around the house and in the farm than boys, so they may not have the same opportunities for studying or even attending primary school. But it is not acceptable to say that girls are less capable of learning mathematics then boys. With the proper attitude and motivation, any student can excel in math.\\

\textbf{Strategies:}
\begin{itemize}
\item\emph{Girls Conferences}: If your region is holding a Girls Conference, be sure to include some sessions on math and science. Teach girls the importance of learning math and the need for female scientists, mathematicians and engineers in Tanzania and the around the world. Have a female businessperson or accountant or doctor come to speak to the girls to reinforce this point from a successful Tanzanian woman's perspective. Give the girls a forum to play math games, enjoy math, and have the freedom to ask any questions they may not feel comfortable asking at school.

\item\emph{Role models}: Teach girls about role models; they should think of someone who they look up to in life and write down all of the characteristics about that person that they want to have themselves. Discuss what characteristics are undesirable in other adults and should not be replicated.

\end{itemize}

\subsection{Teachers give up / No motivation for math teachers}
After years and years of poor test performance in math by students, other math teachers at your school may become discouraged and claim that there is no motivation for math teachers in Tanzania. Do not let them give up so easily! There are always ways to have a positive impact on the students' lives and make improvements in math.\\

\textbf{Strategies:}
\begin{itemize}
\item\emph{Secondary activities}: Involve your fellow math teachers in secondary activities, such as Math Club, Math Day and Math Conferences and Seminars. Let them see the enjoyment students can get from doing math-related games and activities so they can see the benefits that a teacher can have on students' lives.

\item\emph{Recognition for good teachers}: You can also suggest that your school give recognition to good teachers. Have students vote for a Teacher of the Month or Year, based on making an extra effort or trying out alternative teaching methods. You can prepare a certificate to give to these teachers in recognition of their efforts. Again, positive attitudes can be just as contagious as negative ones!

\end{itemize}