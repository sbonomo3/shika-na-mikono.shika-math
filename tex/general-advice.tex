\chapter{General Advice for Math Teachers}
Every school, site, and volunteer experience is different. Variables such as class\slash school size, quality and quantity of past math teachers, available resources, location (rural\slash urban) and school type (day\slash boarding, government\slash private) can all influence a volunteer's teaching approach and experience, so take the tips in this book as a guide only, and not a recipe for teaching success. It is advisable to take plenty of time and lots of observation of your school, classrooms, and students to see what works best for you and your students.
\section{Assessment}
Upon arriving at your school, you may find it helpful to do an overall assessment of the ``math situation'' present. It's good to get a feel for the current status of your school's math department, as well as the attitudes towards mathematics from the students' and teachers' perspectives.

You may come to a school that has never had a math teacher, or hasn't had a math teacher in a few years, or has a math teacher that has not taught regularly. As a result, many of your students may be behind in the syllabus, and it is helpful for you to know what they have and have not learned (or have and have not retained).

Before you start teaching, it may be helpful to test all of your students to see where their math skills are. For example, test your Form I's for basic math skills they should have obtained in primary school. For Form II's, give them questions from their Form I math textbook, and so on. Make the test short and simple, as it will likely take them much more time to complete than you expect! The results will show, in the very least, what topics you will need to spend time reviewing. 

If you are teaching Form III's and IV's, it may also be helpful to look at math scores on the Form II NECTA exam. These will be helpful both in identifying students with strong math skills as well as the weaker students, and will give you a baseline to monitor student improvement later on. This may also show if there are opportunities for you to tailor your classes to the students' pace and ability. For example, if your school has more than one stream, ask your headmaster if they can be separated based on math ability; this will allow you to move at a faster pace with students that are good at math and spend more time with those that pick up concepts more slowly. If students are very far behind and it is possible to redistribute them out of form for math lessons, that can also be effective (although this requires a lot more effort to rework the school timetable). 
	
\section{Time, Time, Time}
You cannot expect Tanzanian students to work as quickly or efficiently as you would hope. Many of them do not have times tables well memorized or basic math in their heads. Unless you plan on drilling them to work on speed, give students plenty of time to write notes, answer questions, and solve problems. They need additional time to translate an answer in their heads from Kiswahili to English, so be patient when you ask a question in class (wait several seconds). Also be very conscious of length when writing an exam. Pre-testing your students is also helpful in gauging how long they take to solve problems before giving your first real exam.
\section{Board Work}
Students love to answer questions they know! You will probably find that they are excited to raise their hands and come solve a problem on the board, but only if they are confident and know the answer. So start with easier questions to build their confidence before going on to more challenging problems. Learn their names, who the bright and weak students are, and call on a variety of them to solve problems. This will also help them stay awake and focused, if they know that they may be called on to solve a problem at any time. Be sure to offer plenty of praise for correctness and help a student work through a problem if they are struggling. You may wish to use your better students to walk through an example on the board in Kiswahili after you've explained it in English. This can be helpful in getting the students to understand a concept, but be cautious of the likely possibility that they will soon stop listening to you and just wait for the Kiswahili explanation! In addition, board races between students are really fun for the students and also a great way to reinforce a topic and increase their problem-solving speed.
\section{Group Work}
The importance of group work cannot be stressed enough! You cannot lecture all period (especially during double periods) and hope to keep their attention. Try to finish the lecture early and give the rest of the class period to work on exercises. Groups of 2-4 students are usually best, and you can count off students to divide them into random groups. It can take time and be chaotic in larger classes as students move to form groups, so try to make a silent game out of it: students race to find and form their group using only hand gestures and mouthing words. 

Group work can be very helpful with the language barrier, as students can discuss and work through problems together in Kiswahili. It is truly exciting to observe a class where several groups are working together and solving problems. Your stronger students can help explain concepts to the slower students, and if one group finishes early, they can go help other groups. You are also there to walk around and help students and check their answers. Many groups will have the same errors, so this is a great time to gauge what topics students understand well or struggle with, and you can help clarify concepts on the board.
\section{Homework}
Your decision of whether or not to assign homework will likely depend on the time available after class for you and your students. If you are teaching at a boarding school, students may have ample time to complete homework in the evenings, but students at day schools (especially girls) may have a lot of work around the house to attend to after school. In addition, your own teaching schedule may or may not permit you to dedicate the time necessary to grade all of the students' notebooks on a regular basis.

As an alternative to assigning homework, you may elect to give students time at the end of class to finish in-class exercises on their own and then go over the answers at the beginning of the next class. Often what you may find after marking an entire class's homework assignment is that the majority of the students either did not bother to complete the assignment or have simply copied answers from one another, which does not give you meaningful feedback as a teacher. Thus, you may find it more effective to be present when giving students exercises to work on, and to gauge student progress and understanding from group work or quizzes and tests.

Whatever you decide in terms of assigning homework, remember: math $=$ practice! The only way for students to gain confidence and succeed in math is to get practice doing exercises \textit{by themselves}! Group work is a great method for improving students' understanding, but they will not have anyone to hold their hands during the NECTA exam, and eventually must be comfortable with their own individual math abilities.

\section{Critical Thinking}
Mathematics is the best subject to teach the process of thinking since math is very logical. Logic is a step
by step process to take given data and to produce the desired results. The step by step process needs to be
taught to the students. If it is taught to students early on in mathematical studies, the students will experience
success in mathematics. Math should not be memorized but processed through thinking. When thinking and processing data, math can be easy and fun.

Critical thinking is basic to real learning, as opposed to pure memorization. It begins early as students understand
that 3 is greater than 2. The student should use a number line or objects to understand this relational concept. The student progresses in critical thinking to more complex problems. Thus, the student can more easily compare or contrast in other subjects if she or he learns to reason in mathematics. There are many math games and races to increase math thinking skills. The process does in fact  need to be taught! Lessons need to be developed that teach students a step by step process of thinking.