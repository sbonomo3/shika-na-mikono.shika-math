\chapter{About This Book}
Hands-On Math is a teaching resource manual for U.S. Peace Corps Volunteers serving in Tanzania. Many of these Volunteers work in schools having insufficient math teachers and teaching resources and all are new to the challenges of teaching in Tanzanian secondary schools. This manual was developed as a guide to assist Volunteers in teaching mathematics in Tanzania more effectively.

Like science, math must be learned by discovery. Mathematics is central to all sciences, and as such requires tangible interaction in order to achieve a full understanding of its concepts. For too long have ancient methods of lecture and rote memorization failed to provide students with a meaningful grasp on the principles of mathematics. In order to inspire students to take ownership of their education, teachers must use any and all available resources to convey the importance, joy and usefulness of learning math. But these resources need to be neither expensive nor difficult to obtain. The vast majority of the teaching aids and activities provided in this book can be utilized at little or no cost and are readily available even in the most rural village settings. This means that students themselves can and should be encouraged to use as many of them as possible so that they can see how the entire world is a forum for math learning.

In addition to fulfilling its own its own academic endeavors, the study of mathematics develops within its students a logical and systematic approach to problem-solving that may be applied to all facets of life. It is in this sense that math infiltrates all fathomable aspects of our lives and affirms itself as a fundamental component in the ongoing process of education. But logic is not necessarily inherent to an individual's mental ability. Instead it must be taught through puzzles, games, stimulating questions and problems applicable to real life. As students gain familiarity taking on certain kinds of problems and identifying suitable approaches to solving them, they increase their capacity for learning and become better adept at independently facing challenges they may come across in the future. Thus, many of the activities presented in this book are not only fruitful in teaching math, but also in developing thinking individuals.

Because many of the challenges faced by math teachers in Tanzania can differ drastically from those in an American context, this book offers some possible strategies for meeting said challenges from a Volunteer's perspective. However, there is no formula for teaching, regardless of location, and so these suggestions should be taken as nothing more than a guide based on the experiences of others. The value of a teacher lies in her ability to address the specific needs of her students, and so any teaching methodology should be open to adaptations and amendments upon further discovery of one's school and environment.

In addition to gaining a familiarity with methods of teaching math in a Tanzanian context, Volunteers must be aware of current exam regulations provided by the Ministry of Education and Vocational Training (MoEVT) in order to properly assist and prepare their students for the NECTA national examinations. To aid the Volunteer, a section has been added to this book outlining the most recently released exam guidelines released by the MoEVT. Additionally, a thorough collection of past NECTA exam problems spanning from 1995 to 2012 has been included and organized according to the current mathematics syllabus of topics from Form I up to Form IV. This is intended to assist the Volunteer in providing relevant examples of problems that students will be accountable to answer on their national examinations.

This book was originally created and published under the enthusiastic inspiration of PCV Marilyn Bick, with meaningful contributions coming from PCVs Cindy del Rosario, Julia Meyers, Teresa Ring and Katherine. Much of the inspiration and labor in converting the original manual into \LaTeX was provided by PCV Dave Berg. Now in its second edition, it has been reorganized in order to better address the needs of Volunteers. Additional content has been added based on inspiring and successful teaching techniques employed by PCVs Jon Mortelette, Sara Tomaskiewicz, John Clay and Belle Archaphorn, among many others. While advancing through their service in Tanzania, Volunteers will likely come across additional resources to contribute in order to further the scope of this book. This is highly encouraged and necessary for its continued relevance in the context of Peace Corps Volunteers teaching math in Tanzania. Volunteers wishing to make contributions, criticisms, concerns or suggestions to this manual can do so using the official Volunteer website of Peace Corps Tanzania (\href{http:/www.pctanzania.org}{pctanzania.org}) or by following the contact information provided below.

In the words of Marilyn Bick,
\begin{center}
\textbf{TEACHING MATHEMATICS IS FUN}\\
\textbf{ENJOY TEACHING}\\
\textbf{INVOLVE THE STUDENTS}\\
\textbf{BE CRAZY SOME DAYS}\\
\textbf{LEARNING WILL HAPPEN}\\
\textbf{NUMBERS ARE TO BE PLAYED WITH DAILY!}\\[24pt]
\end{center}
\begin{tabular}{p{0.5\textwidth}p{0.5\textwidth}}
\multicolumn{1}{l}{Steve Bonomo} & \multicolumn{1}{r}{Marilyn Bick}\\
\multicolumn{1}{l}{Wilima Secondary School, Ruvuma} & \multicolumn{1}{r}{marilynbick@peoplepc.com}\\
\multicolumn{1}{l}{sbonomo3@gmail.com} & 
\end{tabular}
%Steve Bonomo\\
%Wilima Secondary School, Ruvuma\\
%sbonomo3@gmail.com\\[24pt]
%Marilyn Bick\\
%marilynbick@peoplepc.com


